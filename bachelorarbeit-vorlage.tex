% Vorlage für eine Bachelorarbeit
% Siehe auch LaTeX-Kurs von Mathematik-Online
% www.mathematik-online.org/kurse
% Anpassungen für die Fakultät für Mathematik
% am KIT durch Klaus Spitzmüller und Roland Schnaubelt Dezember 2011

\documentclass[12pt,a4paper]{scrartcl}
% scrartcl ist eine abgeleitete Artikel-Klasse im Koma-Skript
% zur Kontrolle des Umbruchs Klassenoption draft verwenden


% die folgenden Packete erlauben den Gebrauch von Umlauten und ß
% in der Latex Datei
\usepackage[utf8]{inputenc}
% \usepackage[latin1]{inputenc} %  Alternativ unter Windows
\usepackage[T1]{fontenc}
%\usepackage[ngerman]{babel}


\usepackage[pdftex]{graphicx}
\usepackage{latexsym}
\usepackage{amsmath,amssymb,amsthm}
\usepackage[arrow, matrix, curve]{xy}
\usepackage{pxfonts}

\usepackage{hyperref}
% Abstand obere Blattkante zur Kopfzeile ist 2.54cm - 15mm
\setlength{\topmargin}{-15mm}


% Umgebungen für Definitionen, Sätze, usw.
% Es werden Sätze, Definitionen etc innerhalb einer Section mit
% 1.1, 1.2 etc durchnummeriert, ebenso die Gleichungen mit (1.1), (1.2) ..
\newtheorem{Theorem}{Theorem}[section]
\newtheorem{Proposition}[Theorem]{Proposition}
\newtheorem{Definition}[Theorem]{Definition} 
\newtheorem{Lemma}[Theorem]{Lemma}		
\newtheorem{Example}[Theorem]{Example}
\newtheorem{Problem}{Problem}           
\newtheorem{Construction}[Theorem]{Construction}           

\numberwithin{equation}{section} 

% einige Abkuerzungen
\newcommand{\C}{\mathbb{C}} % komplexe
\newcommand{\K}{\mathbb{K}} % komplexe
\newcommand{\R}{\mathbb{R}} % reelle
\newcommand{\Q}{\mathbb{Q}} % rationale
\newcommand{\Z}{\mathbb{Z}} % ganze
\newcommand{\N}{\mathbb{N}} % natuerliche
\newcommand{\Pcomplexity}{\mathbf{P}}
\newcommand{\NPcomplexity}{\mathbf{NP}}

\DeclareSymbolFont{symbolsC}{U}{pxsyc}{m}{n}
\DeclareMathSymbol{\coloneqq}{\mathrel}{symbolsC}{"42}


\begin{document}
  % Keine Seitenzahlen im Vorspann
  \pagestyle{empty}

  % Titelblatt der Arbeit
  \begin{titlepage}

    \includegraphics[scale=0.45]{kit-logo.jpg} 
    \vspace*{2cm} 

 \begin{center} \large 
    
    Bachelorarbeit
    \vspace*{2cm}

    {\huge Titel der Bachelorarbeit}
    \vspace*{2.5cm}

    Corvin Paul
    \vspace*{1.5cm}

    Datum der Abgabe
    \vspace*{4.5cm}


    Betreuung: Name der Betreuerin / des Betreuers \\[1cm]
    Fakultät für Mathematik \\[1cm]
		Karlsruher Institut für Technologie
  \end{center}
\end{titlepage}



  % Inhaltsverzeichnis
  \tableofcontents

\newpage
 


  % Ab sofort Seitenzahlen in der Kopfzeile anzeigen
  \pagestyle{headings}

\section{Introduction}

\subsection{Basics in computational complexity theory}
At first we want to introduce some basic notions in complexity theory which will be needed to state the main results of this thesis.
Readers who are familiar with computational complexity theory may want to skip this section.

Computational complexity theory is mainly about classifying under which restrictions problems can be algorithmically solved.
Let's begin with defining what a problem is :

\begin{Definition}
 Let $f \colon {\lbrace 0,1 \rbrace}^* \to {\lbrace 0,1 \rbrace}$ be a function and $L_f \coloneqq {\lbrace 
 x \in {\lbrace 0,1 \rbrace}^*  \; | \; f(x) = 1 \rbrace} $ its corresponding language. We call the 
 problem to decide if a given $x \in {\lbrace 0,1 \rbrace}^*$ lies in $L_f$ the \emph{decision problem} for $f$.
 Or more generally given a language $L \subseteq {\lbrace 0,1 \rbrace}^*$ the decision problem for $L$ is to decide
 if for a given $x \in {\lbrace 0,1 \rbrace}^* $ , $x \in L$ holds.
\end{Definition}

This definition is somehow impractical and one should think of elements $x \in {\lbrace 0,1 \rbrace}^*$ as binary encoding
of some information. It's useful to introduce decision problems like this because we then have a common basis how problems
are represented. However we will implicitly think of problems being encoded as bit strings in the following. Now
we want to look at a first problem, namely $k$-COLOUR. Informally it can be stated like this : \\
 Given an undirected graph $ G = (V,E)$ is it possible to assign a colour (chosen from $k$ different colours) to every vertex 
 $v \in V$ such that adjacent vertices have different colours? In more mathematical language it looks like this:

\begin{Problem}{$k$-COLOUR}
 For $k \in \N$ we define $$L_{k-COLOUR} \coloneqq \lbrace G  = (V,E) \; | \; \text{there is} \; f \colon V \to \lbrace 1, \dotsc k \rbrace
 \; \text{with} \; f(i) \neq f(j) \; \text{for} \; (i,j) \in E \rbrace $$
 
 Then the decision problem for $L_{k-COLOUR}$ is called \emph{$k$-COLOUR}
\end{Problem}

Next we want to classify the ``hardness'' of decision problems and therefore introduce complexity classes. A \emph{complexity class}
 is a set of languages (or functions) that can be computed under certain restrictions. We only want to introduce two 
 of the most important complexity classes namely $\Pcomplexity$ and $\NPcomplexity$. We will informally use the term algorithm
 to mean a deterministic Turing maschine (refer to \cite{Arora2009} chapter 1 for a more formal treatment). 
 
\begin{Definition}
 The class $\Pcomplexity$ consists of those problems $f \colon {\lbrace 0,1 \rbrace}^* \to {\lbrace 0,1 \rbrace}$ for
 which there exists an algorithm $A$ which given an input $x \in {\lbrace 0,1 \rbrace}^*$
 computes $f(x)$ and whose running time can be bounded by a fixed polynomial $p$ (i.e.
 for every input $x \in {\lbrace 0,1 \rbrace}^*$ A runs for at most $p(|x|)$ steps where $|x|$ denotes the length of $x$).
\end{Definition}

We could now think of a complexity class in which every algorithm gets a witness (or proof) that a given input is in the language.
This leads to :

\begin{Definition}
 The class $\NPcomplexity$ consists of those $f \colon {\lbrace 0,1 \rbrace}^* \to {\lbrace 0,1 \rbrace}$ for which there is
 an algorithm $A$ whose running time is polynomially bounded and that can verify for two given inputs $x \in {\lbrace 0,1 \rbrace}^*$ and 
 $w \in {\lbrace 0,1 \rbrace}^*$ (where $|w|$ is bounded by a fixed polynomial in the length of $x$) if $w$ is a proof
 that $f(x) = 1$.
\end{Definition}

To illustrate how one should think of such a ``proof''  we will show the following :

\begin{Proposition}
 $k$-COLOUR $\in \NPcomplexity$.
\end{Proposition}

\begin{proof}
 To prove this we describe an algorithm : \
 If we are given an input $x$ which encodes an undirected graph $G = (V,E)$ and a witness $w$ which encodes a 
 function $\varphi \colon V \to \lbrace 1, \dotsc, k \rbrace$ we accept iff for every $(i,j) \in E$  $\varphi(i) \neq \varphi(j)$ holds.
 This algorithm clearly runs in linear time in the number of edges and accepts iff $w$ encoded a  colouring for $G$ or in other words if
 $w$ was a proof that $x$ lies in $k$-COLOUR.
\end{proof}

Next we want to classify how hard some problems are in comparison with other problems.

\begin{Definition}
 We say that a language $A \subseteq {\lbrace 0,1 \rbrace}^*$ is \emph{polynomial time reducible} to a language 
 $B \subseteq {\lbrace 0,1 \rbrace}^*$ if there is a polynomial time computable function 
 ${\varphi \colon {\lbrace 0,1 \rbrace}^* \to {\lbrace 0,1 \rbrace}^*}$ such that for every $x \in {\lbrace 0,1 \rbrace}^*$ 
 it holds that $x \in A$ iff $\varphi (x) \in B$. We then denote this by $ A \leq_p B$. \\
 A language $B$ is called \emph{$\NPcomplexity$-hard} if $A \leq_p B$ for every $A \in \NPcomplexity$. If $B$ is $\NPcomplexity$-hard
 and $B \in \NPcomplexity$ we also say that $B$ is \emph{$\NPcomplexity$-complete}.
\end{Definition}

To illustrate how polynomial reductions look like we prove the following :

\begin{Theorem}
  $k$-COLOUR is $\NPcomplexity$-complete for $k \geq 3$.
\end{Theorem}

\begin{proof}
 We will show this by induction over $k$ .
 The base case $k = 3$ is essentially the hardest part in the proof and we refer the reader to \cite{Rothe2005}.
 Hence we only have to reduce $k$-COLOUR to $(k+1)$-COLOUR. The reduction works as follows :
 Given a graph $G = (V,E)$ we construct a new graph $G' = (V', E')$ where $V' \coloneqq V \cup \lbrace v_{new} \rbrace$ with
 $v_{new} \notin V$ and $E' \coloneqq E \cup \lbrace (v_{new}, v) \; | \; v \in V \rbrace$, i.e. we just add one new vertex to the graph
 and connect it with every existing vertex. One can now verify that $V$ is $k$-colourable iff $V'$ is $(k+1)$-colourable and the
 construction of $V'$ is linear in the number of vertices.
 
\end{proof}

 
\subsection{Graded Algebras}

In this section $R$ will always be a ring. If not stated otherwise ``module'' will
always mean $R$-module and ``linear'' will mean $R$-linear.

First we want to define some basic algebraic structures which will allow us to define our main algebraic tool
in this thesis, namely Sullivan Algebras.

\begin{Definition}{Graded modules and maps of graded modules}

A \emph{graded module} $M$ consists of a collection of modules $(M^i)_{i \in \Z}$, we will also write
$M = \bigcup_{i \in \Z} M^i$. We call an element 
$m_i \in M^i$ an \emph{element of degree $i$} and write $|m_i| = deg (m_i) = i$. Accordingly a \emph{submodule} $N$ of
a graded module $M$ is a collection of modules $(N^i)_{i \in \Z}$ such that $N^i$ is a submodule of $M^i$. \
Likewise direct sums, quotients, kernels and images are defined degreewise for graded modules.
A \emph{linear map of degree k} $f \colon M \to N$ of graded modules $M$ and $N$ is a collection $(f_i)_{i \in \Z}$ of
linear maps $f_i \colon M^i \to N^{i + k}$. If we leave out the degree, we mean a map of degree $0$.
\end{Definition}

We will say that a graded module $M \coloneqq \bigcup_i M^i$ has a property $P$ (e.g. free) if every module $M^i$ has the property $P$.
 
\begin{Definition}{Differential graded modules}

For a graded module $M$ we call a linear map $d \colon M \to M$ of degree $+1$ which satisfies $ d^2 = 0$ a \emph{differential}.
Then we also call the pair $(M , d)$ a \emph{differential graded module} or a \emph{complex}.
A morphism $f \colon M \to N$ of differential graded modules $(M,d)$ and $(N,d')$ is a linear map of graded modules that commutes with the 
differentials, i.e. the diagram \\

\xymatrix{
\dotsc \ar[r]^{d} & M^{i-1} \ar[d]^{f^{i-1}} \ar[r]^d & M^i \ar[d]^{f^i} \ar[r]^d &M^{i+1}\ar[d]^{f^{i+1}} \ar[r]^{d} & \dotsc\\
\dotsc \ar[r]^{d'} & N^{i-1} \ar[r]^{d'} & N^{i} \ar[r]^{d'} & N^{i+1} \ar[r]^{d'} & \dotsc}

commutes.
\end{Definition}

Often we will be interested in complexes $M$ with $M^i = 0$ for all $i < 0$, we call such complexes \emph{cochain complexes}.
\begin{Definition}
The \emph{homology} of a differential graded module $(M,d)$ is defined as \newline ${H(M) \coloneqq 
H(M,d) \coloneqq ker \, d / im \, d}$.
We call elements of $ker \, d$ \emph{cocycles} and elements of $im \, d$ \emph{coboundaries}.
\end{Definition}

A morphism $f \colon M \to N$ of differential graded modules commutes with the differentials and hence maps cocycles to cocycles
and coboundaries to coboundaries. Therefore it induces a well defined map $H(f) \colon H(M) \to H(N) \; \; [x] \mapsto [f(x)]$. 

Most of the times we will mainly be interested in the homology of a chain complex. Therefore we will sometimes want to replace
a cochain complex with a somehow simpler cochain complex that has the same homology. This motivates the following definition:

\begin{Definition}
 A morphism $f$ of differential graded modules that induces isomorphisms $H^i(f)$ for all $i \in \Z \;$ is called \emph{quasi-isomorphism} and will
 be denoted by ``$\overset{\simeq}{\longrightarrow}$''. \newline
 We call two differential graded modules $(M,d)$ and $(N,d)$ \emph{weakly equivalent} if there's a finite chain of quasi-isomorphisms of differential
 graded modules
 $$ (M,d) \overset{\simeq}{\leftarrow} (M_1,d) \overset{\simeq}{\rightarrow} \cdots 
 \overset{\simeq}{\leftarrow} (M_n,d) \overset{\simeq}{\rightarrow} (N,d)$$
\end{Definition}

\begin{Example}
 Hier sollte singuläre Kohomologie stehen.
\end{Example}

Now that we have modules, we want to enrich their structure with a multiplication and get to graded algebras.

\begin{Definition}
 Given two graded modules $V$ and $W$ we define their \emph{tensor product} \newline
 ${(V \otimes W) = \bigcup_{n \in \Z} (V \otimes W)^n }$ by 
 $$ (V \otimes W)^n \coloneqq \bigoplus_{p + q = n} V^p \otimes W^q$$
 
 and for linear maps $f \colon V \to V'$ of degree p and $g \colon W \to W'$ of degree q we define the induced linear map
 of degree p+q \;  ${f \otimes g \colon V \otimes W \to V' \otimes W'}$ by
 $$ (f \otimes g) ( x \otimes y) = (-1)^{deg \, x \; deg \,y} f(x) \otimes g(y) $$

 A multiplication on a differential graded module M is an associative linear map \newline
 ${m \colon M \otimes M \to M }$ that has a neutral element $1_M \in M^0$, i.e. $m(x \otimes 1_M) = m(1_M \otimes x) = x$.
 In the following we will just write $xy \coloneqq m(x \otimes y)$.
 \end{Definition}

 Given this information the reader probably can guess how a graded algebra could look like and here is the definition:
\begin{Definition}
 A graded module $A$ with a multiplication is called a \emph{graded algebra}. We say that $A$ is \emph{commutative} if
 $xy = (-1)^{deg \, x \; deg \, y} yx$ for all $x,y \in A$ holds. Furthermore a \emph{morphism of graded algebras} 
 $f \colon A \to B$ is a morphism of graded modules that respects multiplication, i.e. 
 $$f(xy) = f(x)f(y) \quad \text{for all} \; x,y \in A \quad \text{and} \; f(1_A) = 1_B $$
\end{Definition}

One should not get confused by the term commutative because it only means commutative in the regular sense in even degrees but
anti commutative if both elements are in odd degrees.
We can also introduce differentials to the world of algebras :
\begin{Definition}
 If $A$ is a (commutative) graded algebra and $d$ is a differential on $A$ that satisfies 
 $$ d(xy) = (dx)y + (-1)^{deg x} x(dy)$$
 then we call $(A,d)$ a \emph{(commutative) differential graded algebra}. If $A^i = 0$ for all $i < 0$ we also
 speak of a \emph{cochain algebra}. \newline
 Given two differential graded algebras $(A,d_1)$ and $(B, d_2)$ we can also make their tensor product a 
 differential graded algebra $(A \otimes B, d)$ by defining 
 $$ d(a \otimes b) \coloneqq d_1(a) \otimes b + (-1)^{deg \, a} a \otimes d_2(b)$$
\end{Definition}

Note that the homology of a (commutative) graded differential algebra is a (commutative) graded algebra.
Next we want to introduce the so called tensor algebra $TV$ of a graded module $V$. It has the nice property
that for every graded module morphism $\varphi \colon V \to A$ to a graded algebra $A$ there's a unique 
graded algebra(!) morphism $\psi \colon TV \to A$ that extends $\varphi$, namely the following diagram commutes :

\xymatrix{
V \ar[dr]^{\varphi} \ar[r] & TV \ar@{-->}[d]^{\exists! \psi} \\
& A
}

%%%%%%%%%%%%%%%%%%%%%%%%%%%%%%%%%
 \newpage  % neuer Abschnitt auf neue Seite, kann auch entfallen
%%%%%%%%%%%%%%%%%%%%%%%%%%%%%%%%%
 
\section{Sullivan algebras}

\begin{Definition}
 Given a free graded module V we call $(\Lambda V, d)$ a \emph{Sullivan algebra} if it has the following properties :
 
  There exists a filtration of graded subspaces $V(0) \subseteq V(1) \subseteq V(2) \subseteq \cdots $ of $V$
  with $\bigcup_{i \in \N_0} V(i) = V$ such that 
  $$ d(V(0)) = 0 \; \text{and} \; d( V(k)) \subseteq  \Lambda V(k-1) $$
  
 Furthermore we call a Sullivan Algebra $(\Lambda V,d)$ \emph{minimal} if $im(d) \subseteq \Lambda^{\geq 2} V$
\end{Definition}

To get a better understanding of this structure we will now present some examples.


 \section{NP-complete problems concerning Sullivan algebras}
 
 In this section we want to show the %TODO Hier NP formatieren!
 $\NPcomplexity$ completeness of certain problems concerning Sullivan algebras. Later we will see that these problems
 translate directly to topological problems. The theorems and proofs in this section mainly follow  \cite{Lechuga2000}.
 
 In the following we want our ground ring $R$ to be  $\Q$, hence we only consider rational Sullivan algebras.
 
 \begin{Definition}
  A Sullivan algebra $(\Lambda V, d)$ is called \emph{elliptic} if both $\pi^*(\Lambda V,d)$ and $H(\Lambda V,d)$ have
  finite dimension.
 \end{Definition}
 
 The first result we want to establish is the following:
 
 \begin{Theorem}
 \label{cohomologyFinTheorem}
  It's a $\NPcomplexity$-hard problem to decide if a given a Sullivan algebra $(\Lambda V,d)$ with $\pi^*(\Lambda V,d)$ finite dimensional 
  is elliptic . %TODO NP formatieren
 \end{Theorem}
 
 
 We will prove this theorem by reducing $k$-COLOUR to deciding if $(\Lambda V,d)$ is elliptic for some special 
 Sullivan algebra $(\Lambda V,d)$ which we will construct as follows : \\
 
 \begin{Construction}
 \label{constructionOfSullivanAlgebra}
 Let $G = (V,E)$ be an undirected, simple, connected, finite graph with vertices $ V = \lbrace v_1, \dotsc , v_n \rbrace $
 and edges $ E = \lbrace (v_r, v_s) \; | \; (r,s) \in J \rbrace$. From this we construct for a given $k \in \N$ the following
 Sullivan algebra $(\Lambda V_{G,k} , d)$ : \\
 
 $ V^{even}_{G,k} \coloneqq \langle x_1, \dotsc , x_n \rangle $ \; with \; $|x_i| = 2$ \; for \; $ i = 1, \dotsc , n$ \; 
 and \; $dx_i \coloneqq 0$ \\
 
 $V^{odd}_{G,k} \coloneqq \langle y_{(r,s)} \rangle$ , $(r,s) \in J$ \; with \; $|y_{(r,s)}| = 2k - 3$ \; and \; $dy_{(r,s)} \coloneqq 
 \sum_{l = 1}^k x_r^{k -l} x_s^{l - 1}$ \\
 
 \end{Construction}

  If $k \geq 3$ the differential of the $y_{(r,s)}$ contains no linear term, hence
  $(\Lambda V_{G,s} ,d)$ is minimal for $k \geq 3$. To prove theorem \ref{cohomologyFinTheorem}, we first need the following lemma:
  
  
\begin{Lemma}
\label{lma:cohomoly+equations}
 $H^*(\Lambda V_{G,k})$ has infinite dimension $\iff$ The system of equations \\
 \begin{equation}
 \label{systemofequations}
 {\lbrace \sum_{l = 1}^k u_r^{k - l} u_s^{l - 1} = 0 \; | \; (r,s) \in J \rbrace}  
 \end{equation}
 
 has a non trivial solution 
 $(\lambda_1 , \dotsc, \lambda_n) \in \C^n$
\end{Lemma}

\begin{proof}
 We use the result of Halperin (\cite{Halperin1988} p. 6) that $H^*(\Lambda V_{G,k})$ has infinite dimension if and only if %TODO Referenz einfügen!
 there's a non trivial morphism of differential graded algebras \\
 ${\varphi \colon (\Lambda V_{G,k},d) \to ( \C [\alpha] ,\acute{d} = 0)}$ with $|\alpha| = 2$. How can such a morphism look like?
 Clearly $\varphi(x_i) = \lambda_i \alpha$ for some $\lambda_i \in \C$ and $\varphi(y_{(r,s)}) = 0$ for all $(r,s) \in J$ , because 
 $(\C [\alpha] , 0)$ is zero in odd degrees. Furthermore $\varphi$ must commute with the differentials, hence 
 for all $(r,s) \in J$
 $$ 0 = \acute{d} (\varphi(y_{(r,s)})) = \varphi(dy_{(r,s)}) = \varphi(\sum_{l = 1}^k x_r^{k -l} x_s^{l - 1})
 = \sum_{l = 1}^k \varphi(x_r)^{k -l} \varphi(x_s)^{l - 1}$$
 
 And this shows that $(\varphi(x_i))_{i = 1, \dotsc , n}$ must be a solution of \ref{systemofequations}, which is not trivial
 if $\varphi$ is not trivial.
 This also works the other way round and every non trivial solution  $(\lambda_1 , \dotsc, \lambda_n)$ of \ref{systemofequations}
 can be used to define a non trivial morphism $\varphi(x_i) = \lambda_i \alpha$.
 \end{proof}

 \begin{Theorem}
   $G = (V,E)$ is k-colourable $\iff$ $(\Lambda V_{(G,k)},d)$ is not elliptic
 \end{Theorem}

 \begin{proof}
  From the construction of $(\Lambda V_{(G,k)},d)$ we see that  $\pi^*(\Lambda V_{(G,k)},d)$ is finite dimensional, therefore
  we only have to care about its cohomology. Suppose now that $G$ is $k$-colourable and we have a colouring
  $f \colon V \to { \lbrace 1, \dotsc , k \rbrace }$ with $f(x_r) \neq f(x_s)$ for $(r,s) \in J$. Let $\zeta_k$ be a primitive 
  $k$-th root of unity, then for $(r,s) \in J$ it holds that
  
  $$ \sum_{l = 1}^k (\zeta_k^{f(x_r)})^{k-l} (\zeta_k^{f(x_s)})^{l-1}
  = \frac{(\zeta_k^{f(x_r)})^{k} - (\zeta_k^{f(x_s)})^{k}}{ \zeta_k^{f(x_r)} - \zeta_k^{f(x_s)}} = 0
  $$
  
  Hence $(\zeta_k^{f(x_i)})_{i = 1, \dotsc, n}$ defines a non trivial solution of \ref{systemofequations}
  and lemma \ref{lma:cohomoly+equations} tells us that $(\Lambda V_{(G,k)},d)$ is not elliptic and this shows
  ``$\implies$''. \\
  Let now $(\Lambda V_{(G,k)},d)$ be not elliptic. Then lemma \ref{lma:cohomoly+equations} gives us a non trivial
  solution $(\lambda_1 , \dotsc, \lambda_n) \in \C^n$ of \ref{systemofequations}. We now use this solution to construct
  a colouring of $G$. First observe that $G$ being connected implies that $\lambda_i \neq 0$ for all $\lambda_i$. Then we have
  for $(r,s) \in J$ that $ \lambda_r^k - \lambda_s^k = ( \lambda_r - \lambda_s) 
  \sum_{l = 1}^k \lambda_r^{k - l} \lambda_s^{l - 1} = 0$ . And since $G$ is connected we have 
  $\lambda_1^k = \dotsc = \lambda_n^k$ , which we can wlog assume to equal $1$. Therefore every $\lambda_i$ is a 
  $k$-th root of unity and we can define $f \colon V \to { \lbrace 1, \dotsc , k \rbrace }$ such that 
  $\lambda_i = \zeta_k^{f(x_i)}$. This $f$ now defines a colouring, because if we assume that for a $(i,j) \in J$
  $f(x_i) = f(x_j)$ holds we also get that 
  $$\sum_{l = 1}^k \lambda_i^{k - l} \lambda_j^{l - 1} = \sum_{l = 1}^k (\zeta_k^{f(x_i)})^{k-l} (\zeta_k^{f(x_j)})^{l-1}
  = \sum_{l = 1}^k (\zeta_k^{f(x_i)})^{k-1} = k (\zeta_k^{f(x_i)})^{k-1} \neq 0$$ 
  which is a contradiction to $(\lambda_1 , \dotsc, \lambda_n)$ being a solution of \ref{systemofequations}.
  
 \end{proof}

 Since the construction \ref{constructionOfSullivanAlgebra} is polynomial this also proves theorem \ref{cohomologyFinTheorem}.
% Literaturverzeichnis (beginnt auf einer ungeraden Seite)
  \newpage
  
\bibliography{bachelorthesis}{}
\bibliographystyle{plain}
 
      
  % ggf. hier Tabelle mit Symbolen 
  % (kann auch auf das Inhaltsverzeichnis folgen)

\newpage
  
 \thispagestyle{empty}


\vspace*{8cm}


\section*{Erklärung}

Hiermit versichere ich, dass ich diese Arbeit selbständig verfasst und keine anderen, als die angegebenen Quellen und Hilfsmittel benutzt, die wörtlich oder inhaltlich übernommenen Stellen als solche kenntlich gemacht und die Satzung des Karlsruher Instituts für Technologie zur Sicherung guter wissenschaftlicher Praxis in der jeweils gültigen Fassung beachtet habe. \\[2ex] 

\noindent
Ort, den Datum\\[5ex]

% Unterschrift (handgeschrieben)



\end{document}

