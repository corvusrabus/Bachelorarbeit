
 \section{From spaces to algebras and back}
 
 \subsection{From spaces to algebras}
 
 \begin{Definition}
  A \emph{simplicial object} $K$ in a category $\mathscr{C}$
  is a collection of objects ${(K_n)}_{n \geq 0}$ in $\mathscr{C}$ with $\mathscr{C}$-morphism
  $\partial_i \colon K_{n+1} \to K_n$ for $i = 0, \ldots, n+1$ called the \emph{face morphisms} and \newline
  $s_i \colon K_n \to K_{n+1}$ for $i = 0, \ldots, n$ called the \emph{degeneracy morphisms}
  
  They must fulfill the following equations : %TODO irgendwie noch schauen ob man das auch horizontal nebeneinander machen kann
  \begin{itemize}
   \item $\partial_i \partial_j = \partial_{j-1} \partial_i$ for $i < j$
   \item  $s_i s_j = s_{j+1} s_j$ for $ i \leq j$
   \item $ \partial_i s_j = \begin{cases}
              s_{j-1} \partial_i 	&\text{if $i < j$} \\
              id 	&\text{if $i = j, j+1$} \\
              s_j \partial_{i-1}	 &\text{if $ i > j + 1$}
             
             \end{cases}
    $	
    \end{itemize}
    As one might expect a map $f \colon K \to L$ of simplicial objects $K$ and $L$ (in the same category $\mathscr{C}$), is a collection
    of $\mathscr{C}$-morphism $( f_n \colon K_n \to L_n)_{ n \geq 0}$ that commute with the face and degeneracy maps, i.e.\
    
  \centerline{
  \xymatrix{
    K_{n+1} \ar[d]^{\partial_i} \ar[r]^{f_{n+1}} & L_{n+1} \ar[d]^{\partial_i} & \text{and} &
    K_n \ar[r]^{f_n} & L_n \\
    K_n \ar[r]^{f_n} & L_n & & 
    K_{n-1} \ar[u]^{s_i} \ar[r]^{f_{n-1}} & L_{n-1} \ar[u]^{s_i}     
    }
  }  
    
    commute for all $i,j,n$. \\
   A simplicial object in the category of sets will also be called a \emph{simplicial set} and a map 
   of simplicial objects between such \emph{simplicial map}, a simplicial object in the category of 
   cochain algebras will be called \emph{simplicial cochain algebra} and so forth.
  
 \end{Definition}

 \begin{Example}
 \label{ex:SingularSimplicesAreSimplicialSets}
    We already have seen an example of a simplicial set, namely the singular simplices 
    $S^{sing}_*(X) \coloneqq (S_n^{sing}(X))_{n \geq 0}$ of a topological space $X$.
    Let
    $\varrho^j \colon \Delta^k \to \Delta^{k - 1}$ ,
    ${\varrho^j(x_0, \dotsc , x_k) \coloneqq (x_0 , \dotsc, x_j + x_{j + 1} , \dotsc, x_k)}$ and 
    $\lambda^i$ as defined in \ref{ex:SingularChainComplex}. We use them to define the face
    and degeneracy maps as follows:
    \begin{align*}
     \partial_i \colon S_{n+1}^{sing}(X) \to S_n^{sing}(X) \quad	 &	 \partial_i(\sigma) =
      \sigma \circ \lambda^i  & \text{for $\sigma \in S_{n+1}^{sing}(X)$} 	\\
     s_i \colon S_n^{sing}(X) \to S_{n+1}^{sing}(X) \quad	&	s_i( \sigma) =
     \sigma \circ \varrho^i  & \text{for $\sigma \in S_{n}^{sing}(X)$}
    \end{align*}
    A calculation shows that they satisfy the corresponding equations.
 \end{Example}

 \begin{Definition}
  Let $K$ be a simplicial set and $A = { \lbrace A_n \rbrace}_{n \geq 0}$ a simplicial cochain algebra, we define the cochain algebra
  $$ A(K) = { \lbrace A^p(K) \rbrace}_{ p \geq 0} $$
  as follows :
  \begin{itemize}
   \item $A^p(K)$ is the set of simplicial maps between $K$ and $A^p$ 
   (where $A^p$ denotes the simplicial set given by $A^p \coloneqq {\lbrace (A_n)^p \rbrace}_{n \geq 0})$, i.e.\ 
   $\phi \in A^p(K)$ sends $\sigma \in K_n$ to $\phi_{\sigma} \in {(A_n)}^p$ and satisfies 
   $\phi_{ \partial_i \sigma} = \partial_i \phi_{\sigma}$ and $\phi_{ s_j \sigma} = s_j \phi_{\sigma}$.
   
   \item Addition, scalar multiplication, differential and multiplication are given by :
   \begin{align*}
    (\phi + \psi)_{\sigma} \coloneqq \phi_{\sigma} + \psi_{\sigma} & &  
    (\lambda \cdot \phi)_{\sigma} \coloneqq \lambda \cdot \phi_{\sigma} \\
    (d \phi)_{\sigma} \coloneqq d( \phi_{\sigma}) & & 
    {(\phi \cdot \psi)}_{\sigma} \coloneqq \phi_{\sigma}  \cdot \psi_{\sigma}
   \end{align*}
  
    \item For a morphism $f \colon A \to B$ \; of simplicial cochain algebras we define the morphism
      ${f(K) \colon A(K) \to B(K)}$ by $(f(K) \phi)_{\sigma} \coloneqq f(\phi_{\sigma})$.
    
    \item For a simplicial map $\varphi \colon K \to L$ we define ${A(\varphi) \colon A(K) \gets A(L)}$
      by $(A(\varphi) \phi)_{\sigma} \coloneqq \phi_{\varphi \sigma}$.
  \end{itemize}
  For a topological space $X$ we shall write $A(X) \coloneqq A(S_*^{sing}(X))$
  %TODO fertig schreiben
 \end{Definition}

 %TODO einleitenden Text für polynomielle Differentialformen schreiben
 
 \begin{Definition}
  The algebra of \emph{polynomial differential forms} $A_{PL} = {\lbrace (A_{PL})_n\rbrace}_{n \geq 0}$ is a
  simplicial object in the category of commutative cochain algebras defined as follows: \newline
  \begin{itemize}
   \item For $n \geq 0$ define the cochain algebra 
      $$(A_{PL})_n \coloneqq \frac{\Lambda \langle t_{0,n}, \dotsc, t_{n,n}, y_{0,n}, \dotsc, y_{n,n} \rangle}
      {((\sum_{i = 0}^n t_{i,n}) - 1, \sum_{i = 0}^n y_{i,n})}
      = \frac{\Lambda \langle t_0, \dotsc, t_n, y_0, \dotsc, y_n \rangle}
      {((\sum_{i = 0}^n t_i) - 1, \sum_{i = 0}^n y_i)} $$
      and specify the differential $d$ by $dt_i = y_i$ and $dy_i = 0$. Later we will drop the first definition with
      double indices and only use the right hand side.
   \item Nevertheless the double indices are useful for defining the face and degeneracy maps 
   $ \partial_i \colon (A_{PL})_{n+1} \to (A_{PL})_n$ and $ s_j \colon (A_{PL})_{n-1} \to (A_{PL})_n$
   as the cochain algebra morphisms induced by:
   
    \begin{center}
      \hfill
      $\!\begin{aligned}[t]
       \partial_i (t_{k,n+1}) \coloneqq \begin{cases}
                                      t_{k,n}  	&\text{for $k < i$} \\
                                      0		&\text{for $k = i$}\\
                                      t_{k-1,n} &\text{for $k > i$}
				      \end{cases}
      \end{aligned}$\hfill\hfill
      $\!\begin{aligned}[t]
       s_j (t_{k,n-1}) \coloneqq \begin{cases}
                                      t_{k,n}  			&\text{for $k < j$} \\
                                      t_{k,n} + t_{k+1,n}	&\text{for $k = i$}\\
                                      t_{k+1,n} 		&\text{for $k > i$}
				      \end{cases}
      \end{aligned}$\hfill
    \end{center}    
  \end{itemize}
%   For a topological space $X$ we shall write $A_{PL}(X) \coloneqq A_{PL}(S_*^{sing}(X))$
 \end{Definition}

 The cochain algebra ${(A_{PL})}_n$ can be seen as a (rational) subalgebra of 
 $\Omega(\Delta^n)$ (the algebra of differential forms on  $\Delta^n$) with coefficients being polynomials in
 the $t_i$ which can be regarded as the coordinate functions of $R^{n+1}$. This explains the name 
 ``algebra of polynomial forms''.
 The next theorem illustrates why the algebra of polynomial differential forms is interesting for us.
 
 \begin{Theorem}
 \label{thm:WeakEqAPL}
  Let $X$ be a topological space, there is a weak equivalence 
  $${C^*_{sing}(X) \overset{\simeq}{\longrightarrow} \ldots \overset{\simeq}{\longleftarrow} A_{PL}(X)}$$
 \end{Theorem}
 
 \begin{Corollary}
  For a topological space $X$ the following holds
  $$ H^*(X) \cong H(A_{PL}(X)) $$
 \end{Corollary}

 Unfortunately, the proof of \ref{thm:WeakEqAPL} is too long to be included here, it can be found
 in \cite{Felix2001} (basically the whole of chapter 10).

 Given this we now see that $A_{PL}(X)$ carries a lot of information about the space $X$. Furthermore, it is
 commutative and for path-connected $X$ has a minimal Sullivan model (\ref{rem:MinimalSullivanModelsExist}). This
 motivates:
 
 \begin{Definition}
  If $X$ is a topological space and $\varphi \colon \Sullivan \to A_{PL}(X)$ is a Sullivan model of $A_{PL}(X)$ 
  then we also refer to this model as a \emph{Sullivan model} for X.
 \end{Definition}

 
 
 \subsection{From algebras to spaces}
 
 Now we want to see how we can can realise a Sullivan algebra $\Sullivan$ as a topological space 
 $| \langle \Sullivan \rangle |$. Moreover, we shall see that $\Sullivan \simeq A_{PL}(| \langle \Sullivan \rangle |)$ for a
 ``sufficient nice'' class of Sullivan algebras $\Sullivan$. \\
 We begin with constructing a contravariant functor that turns commutative cochain algebras into simplicial sets.
 
 \begin{Definition}
  Given a commutative cochain algebra $(A,d)$ we construct the simplicial set 
  $\langle A , d \rangle = {\lbrace (\langle A , d \rangle)_n \rbrace}_{n \geq 0}$ as follows : \\
  \begin{itemize}
   \item $(\langle A,d \rangle)_n$ consists of the morphisms of differential graded algebras 
   $\varphi \colon (A,d) \to (A_{PL})_n$.
  \item The face and degeneracy morphisms are defined by $\partial_i \varphi \coloneqq \partial_i \circ \varphi$ and
  accordingly ${s_j \varphi \coloneqq s_j \circ \varphi}$ 
  \end{itemize}

  Additionally, for a morphism of cochain algebras ${f \colon (A,d) \to (B,d)}$ we define a simplicial map
  ${\langle f \rangle \colon \langle (B,d) \rangle \to \langle (A,d) \rangle}$ by
  ${\varphi \mapsto \varphi \circ f}$. \newline
  This defines a contravariant functor from the category of commutative cochain algebras to the category of simplicial sets
  and is called the \emph{Sullivan realisation}.
 \end{Definition}

 Going one step further we now want to make topological spaces out of simplicial sets.
 
 \begin{Definition}
  Given a simplicial set $K = {\lbrace K_n \rbrace}_{n \geq 0}$ we equip each $K_n$ 
  with the discrete topology and define a topological space $|K|$ by :
  
  $$|K| \coloneqq (\coprod_{n \geq 0} K_n \times \Delta^n) / \sim$$
  
  where $\sim$ is the equivalence relation induced by the relations
  \begin{align*}
   s_j \sigma_n \times x \sim \sigma_n \times \varrho^j x & & \text{for $\sigma_n \in K_n$ , $x \in \Delta^{n+1}$} \\
   \partial_i \sigma_{n+1} \times x \sim \sigma_{n+1} \times \lambda^j x & & \text{for $\sigma_{n+1} \in K_{n+1}$ , $x \in \Delta^n$}
  \end{align*}

   
  for the maps $\varrho^j$ and $\lambda^i$ from~\ref{ex:SingularSimplicesAreSimplicialSets}.
  
  For a simplicial map $f \colon K \to L$ the map $|f| \colon |K| \to |L|$ is defined by
  $$ [(\sigma_n , x)] \mapsto [(f(\sigma_n) , x)] \quad $$
  
  It is welldefined and continous.
  Combined this yields a functor from the category of simplicial sets to the one of topological spaces and
  is called the \emph{Milnor realisation functor}.
 \end{Definition}
 One can show that the Milnor realisation functor always yields a CW-Complex (\cite{Milnor1957}).
 Given this we are now able to construct a functor from commutative cochain algebras to topological spaces.
 
 \begin{Definition}
  Given a commutative cochain algebra $(A,d)$ we call $| \langle (A,d) \rangle |$ its \emph{spatial realisation} and for
  a morphism $\varphi \colon (A,d) \to (B,d)$ we call $|\langle \varphi \rangle|$ its spatial realisation.
 \end{Definition}

 As one might expect, we can find $\Sullivan$ in the structure of  $| \langle \Sullivan \rangle |$ for ``nice'' 
 instances of $\Sullivan$ and this is the next Theorem:
 
 \begin{Theorem}
  Let $\Sullivan$ be a simply connected Sullivan Algebra with $H^n(\Sullivan)$ finite dimensional for all $n \in \N$.
  There is a quasi isomorphism
  $$m_V \colon \Sullivan \to A_{PL}(| \langle \Sullivan \rangle |)$$
 \end{Theorem}

 \begin{proof}
 See \cite{Felix2001} p. 250 Theorem 17.10 .
  
 \end{proof}

 
 