\section{Introduction}

\subsection{Basics in computational complexity theory}
At first we want to introduce some basic notions in complexity theory which will be needed to state the main results of this thesis.
Readers who are familiar with computational complexity theory may want to skip this section.

Computational complexity theory is mainly about classifying under which restrictions problems can be algorithmically solved.
Let us begin with defining what a problem is:

\begin{Definition}
 Let $f \colon {\lbrace 0,1 \rbrace}^* \to {\lbrace 0,1 \rbrace}$ be a function and $L_f \coloneqq {\lbrace 
 x \in {\lbrace 0,1 \rbrace}^*  \; | \; f(x) = 1 \rbrace} $ its corresponding language. We call the 
 problem to decide if a given $x \in {\lbrace 0,1 \rbrace}^*$ lies in $L_f$ the \emph{decision problem} for $f$.
 Or more generally given a language $L \subseteq {\lbrace 0,1 \rbrace}^*$ the decision problem for $L$ is to decide
 if for a given $x \in {\lbrace 0,1 \rbrace}^* $ , $x \in L$ holds.
\end{Definition}

This definition is somehow impractical and one should think of elements $x \in {\lbrace 0,1 \rbrace}^*$ as binary encoding
of some information. It is useful to introduce decision problems like this because it provides a common basis how problems
are represented. However we will implicitly think of problems being encoded as bit strings in the following. Now
we want to look at a first problem, namely $k$-COLOUR. It can be stated like this: \\
 Given an undirected graph $ G = (V,E)$ is it possible to assign a colour (chosen from $k$ different colours) to every vertex 
 $v \in V$ such that adjacent vertices have different colours? In more mathematical language it looks like this:

\begin{Problem}{$k$-COLOUR}
 For $k \in \N$ we define $$L_{k-COLOUR} \coloneqq \lbrace G  = (V,E) \; | \; \text{there is} \; f \colon V \to \lbrace 1, \dotsc k \rbrace
 \; \text{with} \; f(i) \neq f(j) \; \text{for} \; (i,j) \in E \rbrace $$
 
 Then the decision problem for $L_{k-COLOUR}$ is called \emph{$k$-COLOUR}.
\end{Problem}

Next we want to classify the ``hardness'' of decision problems and therefore introduce complexity classes. A \emph{complexity class}
 is a set of languages (or functions) that can be computed under certain restrictions. We only want to introduce two 
 of the most important complexity classes, namely $\Pcomplexity$ and $\NPcomplexity$. We will use the term algorithm
 to mean a deterministic Turing maschine (refer to \cite{Arora2009} chapter 1 for a more formal treatment). 
 
\begin{Definition}
 The class $\Pcomplexity$ consists of those problems $f \colon {\lbrace 0,1 \rbrace}^* \to {\lbrace 0,1 \rbrace}$ for
 which there exists an algorithm $A$ which given an input $x \in {\lbrace 0,1 \rbrace}^*$
 computes $f(x)$ and whose running time can be bounded by a fixed polynomial $p$ (i.e.\
 for every input $x \in {\lbrace 0,1 \rbrace}^*$ A runs for at most $p(|x|)$ steps where $|x|$ denotes the length of $x$).
\end{Definition}

We could now think of a complexity class in which every algorithm gets a witness (or proof) that a given input is in the language.
This leads to:

\begin{Definition}
 The class $\NPcomplexity$ consists of those $f \colon {\lbrace 0,1 \rbrace}^* \to {\lbrace 0,1 \rbrace}$ for which there is
 an algorithm $A$ that can verify for two given inputs $x \in {\lbrace 0,1 \rbrace}^*$ and 
 $w \in {\lbrace 0,1 \rbrace}^*$ (where $|w|$ is bounded by a fixed polynomial in the length of $x$) if $w$ is a proof
 that $f(x) = 1$ and whose running time is polynomially bounded in $|x|$.
\end{Definition}

To illustrate how one should think of such a ``proof''  we will show the following:

\begin{Proposition}
 $k$-COLOUR $\in \NPcomplexity$.
\end{Proposition}

\begin{proof}
 To prove this we describe an algorithm: \
 If we are given an input $x$ which encodes an undirected graph $G = (V,E)$ and a witness $w$ which encodes a 
 function $\varphi \colon V \to \lbrace 1, \dotsc, k \rbrace$ we accept iff $\varphi(i) \neq \varphi(j)$ holds
 for every $(i,j) \in E$ .
 This algorithm clearly runs in linear time in the number of edges and accepts iff $w$ encoded a  colouring for $G$ or in other words if
 $w$ was a proof that $x$ lies in $k$-COLOUR.
\end{proof}

Next we want to classify how hard some problems are in comparison with other problems.

\begin{Definition}
 We say that a language $A \subseteq {\lbrace 0,1 \rbrace}^*$ is \emph{polynomial time reducible} to a language 
 $B \subseteq {\lbrace 0,1 \rbrace}^*$ if there is a polynomial time computable function 
 ${\varphi \colon {\lbrace 0,1 \rbrace}^* \to {\lbrace 0,1 \rbrace}^*}$ such that for every $x \in {\lbrace 0,1 \rbrace}^*$ 
 it holds that $x \in A$ iff $\varphi (x) \in B$. We then denote this by $ A \leq_p B$. \\
 A language $B$ is called \emph{$\NPcomplexity$-hard} if $A \leq_p B$ for every $A \in \NPcomplexity$. If $B$ is $\NPcomplexity$-hard
 and $B \in \NPcomplexity$ we also say that $B$ is \emph{$\NPcomplexity$-complete}.
\end{Definition}

\begin{Remark}
 Note that there are $\NPcomplexity$-hard problems which do not lie in $\NPcomplexity$ (e.g. the halting problem)
 and therefore introducing the term $\NPcomplexity$-complete is a useful convention.
\end{Remark}


To illustrate how polynomial reductions look like we prove the following:

\begin{Theorem}
  $k$-COLOUR is $\NPcomplexity$-complete for $k \geq 3$.
\end{Theorem}

\begin{proof}
 We will show this by induction over $k$.
 The base case $k = 3$ is essentially the hardest part in the proof and we refer to \cite{JorgRothe2008} (Theorem 3.56).
 Hence we only have to reduce $k$-COLOUR to $(k+1)$-COLOUR. The reduction works as follows: \newline
 Given a graph $G = (V,E)$ we construct a new graph $G' = (V', E')$ where $V' \coloneqq V \cup \lbrace v_{new} \rbrace$ with
 $v_{new} \notin V$ and $E' \coloneqq E \cup \lbrace (v_{new}, v) \; | \; v \in V \rbrace$, i.e.\ we just add one new vertex to the graph
 and connect it with every existing vertex. One can now verify that $V$ is $k$-colourable iff $V'$ is $(k+1)$-colourable and the
 construction of $V'$ is linear in the number of vertices.
 
\end{proof}

 
\subsection{Graded Algebras}

In this section $R$ will always be a ring. If not stated otherwise ``module'' will
always mean $R$-module, ``algebra'' will mean $R$-algebra and ``linear'' will mean $R$-linear.

First we want to define some basic algebraic structures which will allow us to define our main algebraic tool
in this thesis, namely Sullivan Algebras.

\begin{Definition}{Graded modules and maps of graded modules}

A \emph{graded module} $M$ consists of a collection of modules $(M^i)_{i \in \Z}$, we will also write
$M = \bigcup_{i \in \Z} M^i$. We call an element 
$m_i \in M^i$ a \emph{homogeneous} element of \emph{degree} $i$ and write $|m_i| = deg (m_i) = i$. Accordingly a \emph{submodule} $N$ of
a graded module $M$ is a collection of modules $(N^i)_{i \in \Z}$ such that $N^i$ is a submodule of $M^i$. \
Likewise direct sums, quotients, kernels and images are defined degreewise for graded modules. \newline
A \emph{linear map of degree k} $f \colon M \to N$ of graded modules $M$ and $N$ is a collection $(f_i)_{i \in \Z}$ of
linear maps $f_i \colon M^i \to N^{i + k}$. If we leave out the degree, we mean a map of degree $0$.
\end{Definition}

We will say that a graded module $M \coloneqq \bigcup_i M^i$ has a property $P$ (e.g. free) if every module $M^i$ has the property $P$.
 
\begin{Example}[Singular chain complex]
\label{ex:SingularChainComplex}
 Let $\Delta^n \coloneqq {\lbrace (x_0, \dotsc , x_n) \in \R^{n+1} \; | \; \sum_{i = 0}^n x_i = 1\rbrace}$ be the standard
 $n$-simplex. Let $X$ be a topological space and denote by
 $S_n^{sing}(X) \coloneqq {\lbrace \sigma \colon \Delta^n \to X \; | \; \sigma \, continous \rbrace}$ the set of
 \emph{singular $n$-simplices}. 
 Now take the free $R$-module with basis $S_n^{sing}(X)$ 
 and denote it by $C_n^{sing}(X;R)$.
 Then $ C_*^{sing}(X;R) \coloneqq (C_n^{sing}(X;R))_{n \geq 0}$ is a free graded module and is called 
 \emph{singular $R$-chain complex} of X. \newline
 For $0 \leq k \leq n+1$ we define the $k$-th face map $\lambda^k \colon \Delta^{n} \to \Delta^{n+1}$ by
 $(x_0, \dotsc, x_n) \mapsto (x_0, \dotsc, x_{k-1},0, x_k, \dotsc, x_n)$.
 Using this we define the map $d \colon C_n^{sing}(X;R) \to C_{n-1}^{sing}(X;R)$ by
 $$ d(\sigma) \coloneqq \sum_{k = 0}^{n+1} (-1)^k \; \sigma \circ \lambda^k$$
 It is a linear map of degree $-1$ and a calculation shows that $d^2 = 0$.
 \end{Example}

\begin{Definition}{Differential graded modules}

For a graded module $M$ we call a linear map $d \colon M \to M$ of degree $+1$ which satisfies $ d^2 = 0$ a \emph{differential}.
Then we also call the pair $(M , d)$ a \emph{differential graded module} or a \emph{complex}.
A morphism $f \colon M \to N$ of differential graded modules $(M,d)$ and $(N,d')$ is a linear map of graded modules that commutes with the 
differentials, i.e.\ the diagram \\

\centerline{
\xymatrix{
\dotsc \ar[r]^{d} & M^{i-1} \ar[d]^{f^{i-1}} \ar[r]^d & M^i \ar[d]^{f^i} \ar[r]^d &M^{i+1}\ar[d]^{f^{i+1}} \ar[r]^{d} & \dotsc\\
\dotsc \ar[r]^{d'} & N^{i-1} \ar[r]^{d'} & N^{i} \ar[r]^{d'} & N^{i+1} \ar[r]^{d'} & \dotsc}
}
commutes.
\end{Definition}

Often we shall be interested in complexes $M$ with $M^i = 0$ for all $i < 0$, we call such complexes \emph{cochain complexes}.

\begin{Example}[Singular cochain complex]
 If we dualise the singular chain complex of a topological space $X$ we obtain its \emph{singular $R$-cochain complex} which 
 is defined as follows: \newline
 $c \coloneqq hom_R(C_n^{sing}(X),R)$ and $C^*_{sing}(X) \coloneqq (C^n_{sing}(X))_{n \geq 0}$. \newline
 Let us denote for a moment the map $d \colon C_{n+1}^{sing}(X) \to C_n^{sing}(X)$ by $\bar{d}$. Then we define the map
 $d \colon C_{sing}^n(X) \to C_{sing}^{n+1}(X)$ by $\phi \mapsto \phi \circ \bar{d}$. Then $\bar{d}^2 = 0$ implies $d^2 = 0$
 hence $d$ is a differential and $(C^*_{sing}(X),d)$ is a cochain complex.
\end{Example}

\begin{Definition}
The \emph{homology} of a differential graded module $(M,d)$ is defined as \newline ${H(M) \coloneqq 
H(M,d) \coloneqq ker \, d / im \, d}$.
We call elements of $ker \, d$ \emph{cocycles} and elements of $im \, d$ \emph{coboundaries}.
Moreover, we call a cochain complex $(M,d)$ \emph{connected} if $H^0(M) \cong R$ and $n$-connected if 
it is connected and $H^i(M) = 0$ for $i = 1, \dotsc, n$. For $1$-connected we shall use the term \emph{simply connected}.
\end{Definition}

A morphism $f \colon M \to N$ of differential graded modules commutes with the differentials and hence maps cocycles to cocycles
and coboundaries to coboundaries. Therefore it induces a well defined map $H(f) \colon H(M) \to H(N) \; \; [x] \mapsto [f(x)]$. 

\begin{Example}[Singular cohomology]
 Given a topological space $X$ we call the homology $H(C^*_{sing}(X;R))$ the \emph{singular cohomology} of the space $X$
 and denote it by $H^*(X;R)$. We also use the convention $H^*(X) \coloneqq H^*(X; \Q)$	
\end{Example}


Most of the times we will mainly be interested in the homology of a chain complex. Therefore we will sometimes want to replace
a cochain complex with a somehow simpler cochain complex that has the same homology. This motivates the following definition:

\begin{Definition}
 A morphism $f$ of differential graded modules that induces isomorphisms $H^i(f)$ for all $i \in \Z \;$ is called a
 \emph{quasi-isomorphism} and will be denoted by ``$\overset{\simeq}{\longrightarrow}$''. \newline
 We call two differential graded modules $(M,d)$ and $(N,d)$ \emph{weakly equivalent} if there is a finite chain of quasi-isomorphisms of differential
 graded modules
 $$ (M,d) \overset{\simeq}{\leftarrow} (M_1,d) \overset{\simeq}{\rightarrow} \cdots 
 \overset{\simeq}{\leftarrow} (M_n,d) \overset{\simeq}{\rightarrow} (N,d)$$
\end{Definition}


Now that we have modules, we want to enrich their structure with a multiplication and get to graded algebras.

\begin{Definition}
 Given two graded modules $V$ and $W$ we define their \emph{tensor product} \newline
 ${(V \otimes W) = \bigcup_{n \in \Z} (V \otimes W)^n }$ by 
 $$ (V \otimes W)^n \coloneqq \bigoplus_{p + q = n} V^p \otimes W^q$$
 By the convention $|v \otimes w| \coloneqq |v| + |w|$ (for homogeneous elements $v \in V$ and $w \in W$
 it also becomes a graded module. \newline
 For linear maps $f \colon V \to V'$ of degree p and $g \colon W \to W'$ of degree q we define the induced linear map
 of degree p+q \;  ${f \otimes g \colon V \otimes W \to V' \otimes W'}$ by
 $$ (f \otimes g) ( x \otimes y) = (-1)^{deg \, x \; deg \,y} f(x) \otimes g(y) $$

 A multiplication on a differential graded module M is an associative linear map \newline
 ${m \colon M \otimes M \to M }$ that has a neutral element $1_M \in M^0$, i.e.\ $m(x \otimes 1_M) = m(1_M \otimes x) = x$.
 In the following we will just write $xy \coloneqq m(x \otimes y)$.
 \end{Definition}

 Given this information the reader probably can guess how a graded algebra could look like and here is the definition:
\begin{Definition}
 A graded module $A$ with a multiplication is called a \emph{graded algebra}. We say that $A$ is \emph{commutative} if
 $xy = (-1)^{deg \, x \; deg \, y} yx$ for all $x,y \in A$ holds. Furthermore, a \emph{morphism of graded algebras} 
 $f \colon A \to B$ is a morphism of graded modules that respects multiplication, i.e.\ 
 $$f(xy) = f(x)f(y) \quad \text{for all} \; x,y \in A \quad \text{and} \; f(1_A) = 1_B $$
\end{Definition}

One should not get confused by the term commutative since it only means commutative in the regular sense in even degrees but
anti commutative if both elements are in odd degrees.
We can also introduce differentials to the world of algebras:
\begin{Definition}
 If $A$ is a (commutative) graded algebra and $d$ is a differential on $A$ that is a \emph{derivation} which means
 that it satisfies 
 $$ d(xy) = (dx)y + (-1)^{deg x} x(dy)$$
 then we call $(A,d)$ a \emph{(commutative) differential graded algebra}. If $A^i = 0$ for all $i < 0$ we also
 speak of a \emph{cochain algebra}. \newline
 Given two differential graded algebras $(A,d_1)$ and $(B, d_2)$ we can make their tensor product a 
 differential graded algebra $(A \otimes B, d)$ by defining 
 $$ d(a \otimes b) \coloneqq d_1(a) \otimes b + (-1)^{deg \, a} a \otimes d_2(b)$$
\end{Definition}

Note that the homology of a (commutative) graded differential algebra is a (commutative) graded algebra. \newline
Next we want to introduce the tensor algebra $TV$ of a graded module $V$. It has the nice property
that for every graded module morphism $\varphi \colon V \to A$ to a graded algebra $A$ there is a unique 
graded algebra(!) morphism $\psi \colon TV \to A$ that extends $\varphi$, namely the following diagram commutes:

\centerline{ \xymatrix{
V \ar[dr]^{\varphi} \ar[r] & TV \ar@{-->}[d]^{\exists! \psi} \\
& A
}
}

\begin{Definition}
 The \emph{tensor algebra} of a graded $R$-module $V$ is given by
 $$ TV \coloneqq \bigoplus_{i = 0}^\infty T^i V$$
 where $T^0 V \coloneqq R$ and $T^i V \coloneqq \bigotimes_{k=1}^i V$. \newline
 As a direct sum of graded modules it is again a graded module and the induced grading on an element
 $ v = \otimes_{i = 1}^k v_i $ (with $v_i \in V$ homogeneous) is given by $ |v| = \sum_{ i = 1}^k |v_i|$.
 For such $v$ we will say that $v$ has \emph{wordlength} $k$.
 Furthermore, it becomes a graded algebra by the multiplication $xy \coloneqq x \otimes y$.
\end{Definition}

Although the tensor algebra is a very nice algebra it has the problem that it is not commutative in general. We can fix this by
defining the free commutative algebra for rings $R$ where $0 \neq 2$.

\begin{Definition}
 Let $V$ be a free graded $R$-module with $0 \neq 2$ in R. Let $I \subset TV$ be the ideal generated by the elements
 $$ v \otimes w - (-1)^{deg \, v \; deg \, w} w \otimes v$$
 for all $v,w \in V$. Then the quotient algebra $TV/I$ is called the \emph{free commutative graded algebra} over $V$.
\end{Definition}

The free commutative graded algebra is always commutative since we divide out all elements which would prevent it 
from being commutative. Furthermore, it has a similar universal property like the tensor algebra: For every graded
module $V$, commutative graded algebra $A$ and linear map $\varphi \colon V \to A$ there is a unique morphism of
graded algebras $\psi \colon \Lambda V \to A$ which makes the following diagram commutative

 

\centerline{\xymatrix{
V \ar[dr]^{\varphi} \ar[r] & \Lambda V \ar@{-->}[d]^{\exists! \psi} \\
& A
}
}



Moreover, we have to introduce some more notation: Given a free graded module $V$, we denote 
by $\Lambda^{\geq k} V$ the vector space of elements in $\Lambda V$ with wordlength at least $k$.
Additionally $V^{\geq k}$ denotes the subspace of elements of degree at least k and ${\Lambda V}^{\geq k}$ 
is the free commutative algebra over this subspace $V^{\geq k}$. Unfortunately these two notations look 
very similar and are very different in their nature, one should not confuse them.

\par

Later it will be important to have a good understanding of free commutative algebras and therefore
we will now provide some examples.

\begin{Example}
\label{ex:FreeCommutativeEvenDegrees}
 To start off simple, we look at $\Lambda \langle v \rangle$ with $|v| = 2$. 
 It is isomorphic to the polynomial algebra $R[v]$.
\end{Example}

\begin{Example}
\label{ex:FreeCommutativeOddDegrees}
 A similiar example is $\Lambda \langle v \rangle$ with $|v| = 3$. Since 
 $$0 = v^2 - (-1)^{deg \, v \; deg \, v} v^2 = 2 v^2$$
 and $2 \neq 0$ we get that $v^2 = 0$. This tells us that all elements in $\Lambda \langle v \rangle$
 have the form $\lambda_1 v + \lambda_2$ for $\lambda_i \in R$ and multiplication is determined by $v^2 = 0$.
\end{Example}

Combining these two behaviours in odd and even degrees we can look at one last example which
illustrates the general behaviour of the free commutative algebra.

\begin{Example}
 Consider the algebra $\Lambda \langle a,b,c \rangle$ with $|a| = 2$ and $|b| = |c| = 3$.
 Analagous to the last example we see that $b^2 = c^2 = 0$. Furthermore,
 $$ bc - (-1)^{deg \, b \; deg \, c} cb = bc + cb = 0$$
 which tells us that $bc = - cb$. This determines the complete multiplicative structure 
 and all elements are of the form 
 $$ \sum_{i,j,k \geq 0} \lambda_{i,j,k} \, a^{p_i} b^{q_j} c^{r_k}$$
 with  $q_j, r_k \in \lbrace 0,1 \rbrace$, $p_i \geq 0$ and $\lambda_{i,j,k} \in R$.
\end{Example}


%TODO mehr Information über freie kommutative Algebra einfügen