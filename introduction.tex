\section*{Introduction}

\newpage

\section{Preliminaries}
\subsection{Computational complexity theory}
We begin with recalling some basic concepts from complexity theory, which are needed to state the main results of this thesis.
Computational complexity theory is mainly concerned with classifying how fast problems can be solved algorithmically
under restrictions.
We get going by defining what a problem is.

\begin{Definition}
 Let $f \colon {\lbrace 0,1 \rbrace}^* \to {\lbrace 0,1 \rbrace}$ be a function and $L_f \coloneqq {\lbrace 
 x \in {\lbrace 0,1 \rbrace}^*  \; | \; f(x) = 1 \rbrace} $ its corresponding language. We call the 
 problem to decide if a given $x \in {\lbrace 0,1 \rbrace}^*$ lies in $L_f$ the \emph{decision problem} for $f$.
 More generally, given a language $L \subseteq {\lbrace 0,1 \rbrace}^*$ the decision problem for $L$ is to decide
 if a given $x \in {\lbrace 0,1 \rbrace}^* $ lies in $L$.
\end{Definition}

\begin{Remark}
The previous definition is somehow impractical, and one should think of elements $x \in {\lbrace 0,1 \rbrace}^*$ as binary encoding
of some information. However, it is useful to introduce decision problems like this since it provides a common basis how problems
are represented. We shall implicitly think of problems as being encoded as bit strings in the following.
\end{Remark}

\begin{Remark}
 We do not have to restrict problems to boolean functions. We could also speak of the complexity of
 computing a function $f \colon {\lbrace 0,1 \rbrace}^* \to {\lbrace 0,1 \rbrace}^*$. 
\end{Remark}


Now we want to have a look at a first concrete problem, namely $k$-COLOUR. It can be stated like this: \\
 Given an undirected graph $ G = (V,E)$, is it possible to assign a colour (chosen from $k$ different colours) to every vertex 
 $v \in V$ such that adjacent vertices have different colours? Or in more formal language:

\begin{Problem}[$k$-COLOUR]
 For $k \in \N$ we define $$L_{k-COLOUR} \coloneqq \lbrace G  = (V,E) \; | 
 \; \text{there is} \; f \colon V \to \lbrace 1, \dotsc k \rbrace
 \ \text{with} \; f(i) \neq f(j) \; \text{for} \; (i,j) \in E \rbrace $$
 
 The decision problem for $L_{k-COLOUR}$ is called \emph{$k$-COLOUR}.
\end{Problem}

Next, we want to classify the hardness of decision problems. Therefore, we introduce complexity classes. 
 In general, a \emph{complexity class}
 is a set of languages (or functions) that can be computed under certain restrictions. We shall only introduce two 
 of the most important complexity classes, namely $\Pcomplexity$ and $\NPcomplexity$. For this, we will use the term algorithm
 to mean a deterministic Turing maschine (refer to~\cite{Arora2009} Chapter 1 for a more formal treatment). 
 
\begin{Definition}
 The class $\Pcomplexity$ consists of those problems $f \colon {\lbrace 0,1 \rbrace}^* \to {\lbrace 0,1 \rbrace}$ for
 which there exists an algorithm $A$ which computes $f(x)$ for a given input $x \in {\lbrace 0,1 \rbrace}^*$, 
 and whose running time can be bounded by a fixed polynomial $p$. That is,
 for every input $x \in {\lbrace 0,1 \rbrace}^*$ $A$ runs for at most $p(|x|)$ steps where $|x|$ denotes the length of $x$.
\end{Definition}


\begin{Definition}
 The class $\NPcomplexity$ consists of those $f \colon {\lbrace 0,1 \rbrace}^* \to {\lbrace 0,1 \rbrace}$ for which there is
 an algorithm $A$ that can verify for two given inputs $x \in {\lbrace 0,1 \rbrace}^*$ and 
 $w \in {\lbrace 0,1 \rbrace}^*$ if $w$ is a proof
 that $f(x) = 1$. Further, the running time of $A$ and $|w|$ must be bounded by a fixed polynomial in  $|x|$.
\end{Definition}


To illustrate how one should think of such a proof, we will show the following proposition.

\begin{Proposition}
 $k$-COLOUR $\in \NPcomplexity$.
\end{Proposition}

\begin{proof}
 To prove this we describe an algorithm. \
 Suppose we are given an input $x$, which encodes an undirected graph $G = (V,E)$, and a witness $w$, which encodes a 
 function $\varphi \colon V \to \lbrace 1, \dotsc, k \rbrace$.  The algorithm accepts if and only if $\varphi(i) \neq \varphi(j)$ holds
 for every $(i,j) \in E$.
 This algorithm clearly runs in linear time in the number of edges. It accepts if and only if
 $w$ encoded a  colouring for $G$, or in other words, if
 $w$ was a proof that $x$ lies in $k$-COLOUR.
\end{proof}

Next, we want to classify how hard some problems are in comparison with other problems.

\begin{Definition}
 A language $A \subseteq {\lbrace 0,1 \rbrace}^*$ is \emph{polynomial time reducible} to a language 
 $B \subseteq {\lbrace 0,1 \rbrace}^*$ if there is a polynomial time computable function 
 ${\varphi \colon {\lbrace 0,1 \rbrace}^* \to {\lbrace 0,1 \rbrace}^*}$ such that $x \in A$ if and only if $\varphi (x) \in B$
 for every $x \in {\lbrace 0,1 \rbrace}^*$. Then, this is denoted by $ A \leq_p B$. 
 
 A language $B$ is called \emph{$\NPcomplexity$-hard} if $A \leq_p B$ for every $A \in \NPcomplexity$. If $B$ is $\NPcomplexity$-hard
 and $B \in \NPcomplexity$, then $B$ is called \emph{$\NPcomplexity$-complete}.
\end{Definition}

\begin{Remark}
 More generally, the problem to compute a function $f \colon {\lbrace 0,1 \rbrace}^* \to {\lbrace 0,1 \rbrace}^*$ shall be called
 $\NPcomplexity$-hard if the computation of $f$ allows the computation of all problems in $\NPcomplexity$ with polynomial overhead.
\end{Remark}

\begin{Remark}
 Note that there are $\NPcomplexity$-hard problems that do not lie in $\NPcomplexity$. 
 An example of this is the halting problem.
 Therefore, introducing the term $\NPcomplexity$-complete is a reasonable convention.
\end{Remark}


To illustrate how polynomial reductions look like, we prove the following theorem.

\begin{Theorem}
  $k$-COLOUR is $\NPcomplexity$-complete for $k \geq 3$.
\end{Theorem}

\begin{proof}
 We already know that $k$-COLOUR $\in \NPcomplexity$, so it remains to show that it is $\NPcomplexity$-hard.
 This will be shown by induction over $k$.
 The base case $k = 3$ is essentially the hardest part in the proof and we refer to~\cite{JorgRothe2008} Theorem 3.56.
 Hence, we only have to reduce $k$-COLOUR to $(k+1)$-COLOUR. The reduction works as follows.
 Suppose we are given an instance of $k$-COLOUR, i.e.\
 a graph $G = (V,E)$. Construct a new graph $G' \coloneqq (V', E')$ where $V' \coloneqq V \cup \lbrace v_{new} \rbrace$ with
 $v_{new} \notin V$ and $E' \coloneqq E \cup \lbrace \lbrace v_{new}, v \rbrace \; | \; v \in V \rbrace$. That is,
 we just add one new vertex to the graph
 and connect it with every existing vertex. We show that $G$ is $k$-colourable if and only if $G'$ is $(k+1)$-colourable.
 Suppose we are given a colouring $\varphi \colon V \to {\lbrace 1,\ldots,k \rbrace}$ of $G$. We define 
 $\varphi' \colon V' \to { \lbrace 1, \ldots, k+1 \rbrace }$
 by 
 $$ \varphi'(v) = \begin{cases}
                   k + 1 &\text{if $v = v_{new}$} \\
                   \varphi(v) &\text{else}
                  \end{cases}
$$
 This defines a $(k+1)$-colouring of $G'$. 
 Otherwise, suppose we are given a $(k+1)$-colouring $\varphi$ of $G'$. By permuting colours, we can assume
 $\varphi (v_{new}) = k+1$. Since every $v \in V$ is connected to $v_{new}$, we have 
 $\varphi(v) \neq \varphi(v_{new}) = k+1$ for every $v \in V$. 
 Thus, the restriction $\restr{\varphi}{V}$ defines a $k$-colouring of $G$.
 
 It remains to see that this reduction runs in polynomial time. This is clear since
 the construction of $G'$ is linear in the number of vertices of $G$.
 
\end{proof}

Later, we will need the following two $\NPcomplexity$-complete problems.

\begin{Problem}[Independent set]
 Let $G = (V,E)$ be a graph. A subset $I \subseteq V$ is called an \emph{independent set} if no two vertices in $I$
 are adjacent. We define the language
  $$L_{INDSET} \coloneqq \lbrace (G,k) \; | \; \text{G is a graph and has an independent set of size k} \rbrace $$
  and call its decision problem \emph{INDSET}.
\end{Problem}

\begin{Problem}[Subset sum]
 We define the language
 $$ L_{SSS} \coloneqq {\big{\lbrace} (x_1, \ldots, x_n, S) \in \N^{n+1} \, | \;
 \exists (b_1, \ldots, b_n) \in {\lbrace 0,1 \rbrace}^n  \; \text{with} \; \sum_{i = 1}^n b_i x_i = S \big{\rbrace}} $$
 and call the decision problem for $L_{SSS}$ SUBSETSUM.
\end{Problem}



\begin{Theorem}
 INDSET and SUBSETSUM are $\NPcomplexity$-complete.
\end{Theorem}

\begin{proof}
 See~\cite{JorgRothe2008} Theorem 3.54 and Theorem 3.67.
\end{proof}

 
 
\subsection{Graded Algebras}

Here, we introduce some algebraic notions of graded modules and algebras, and their properties. At the end of 
this section we want to be able to understand the structure of free commutative algebras. They are the main ingredient 
for defining Sullivan algebras, the most important algebraic structure of this thesis.

In this section $R$ will always be a ring, module will
always mean $R$-module, algebra will mean $R$-algebra, and linear will mean $R$-linear.


\begin{Definition}[Graded modules and maps of graded modules]

A \emph{graded module} $M$ consists of a collection of modules $ {\lbrace M^i \rbrace}_{i \in \Z}$. We shall also write
$M = \bigcup_{i \in \Z} M^i$. An element 
$m_i \in M^i$ is called a \emph{homogeneous} element of \emph{degree} $i$, and we write $|m_i| = deg (m_i) = i$. 
Accordingly, a \emph{submodule} $N$ of
a graded module $M$ is a collection of modules ${\lbrace N^i \rbrace}_{i \in \Z}$ such that $N^i$ is a submodule of $M^i$. \
Likewise, direct sums, quotients, kernels, and images are defined degreewise for graded modules. 

A \emph{linear map}  of graded modules  $f \colon M \to N$ \emph{of degree k} is a collection 
${\lbrace f_i \rbrace}_{i \in \Z}$ of
linear maps $f_i \colon M^i \to N^{i + k}$. If we leave out the degree, we shall mean a map of degree $0$.
\end{Definition}

For a graded module $M$, we shall denote the elements of $M$ that lie in odd degrees by $M^{odd}$ and 
likewise the elements in even degrees by $M^{even}$.
Further, we shall say that a graded module $M \coloneqq \bigcup_i M^i$ has a property $P$ (e.g.\ free) if every module $M^i$ has the property $P$.
 
\begin{Example}
\label{ex:PolynomialRingGradedModule}
 A simple example of a graded module is a polynomial ring $R[X]$. We set the degree of $X$ to be $|X| \coloneqq i$ for $i \in \N$.
 Then, we extend the degree of $X$ to $|X^n| \coloneqq n i$. If $i = 1$, this corresponds to the regular degree of monomials.
 
 Note that $R[X]$ would be a graded module with the grading $|X^i| = 1$ for all $i \in \N$. This grading 
 seems unnatural but is allowed by the definition of graded modules.
\end{Example}
 
 
\begin{Example}[Singular chain complex]
\label{ex:SingularChainComplex}
 Let $\Delta^n \coloneqq \lbrace (x_0, \dotsc , x_n) \in \R^{n+1} \; | \; \sum_{i = 0}^n x_i = 1\rbrace$ be the standard
 $n$-simplex. Let $X$ be a topological space and denote by
 $S_n^{sing}(X) \coloneqq \lbrace \sigma \colon \Delta^n \to X \; | \; \sigma \, continous \rbrace$ the set of
 \emph{singular $n$-simplices}. 
 Take the free $R$-module with basis $S_n^{sing}(X)$ 
 and denote it by $C_n^{sing}(X;R)$.
 Then, $ C_*^{sing}(X;R) \coloneqq {\lbrace C_n^{sing}(X;R)\rbrace}_{n \geq 0}$ is a free graded module and is called 
 \emph{singular $R$-chain complex} of X. 
 
 For $0 \leq k \leq n+1$, define the $k$-th face map
 
 \begin{align*}
 \lambda^k \colon \Delta^{n} \to \Delta^{n+1}&  &
 (x_0, \dotsc, x_n) \mapsto (x_0, \dotsc, x_{k-1},0, x_k, \dotsc, x_n).
 \end{align*}
 Using this, define the map $d \colon C_n^{sing}(X;R) \to C_{n-1}^{sing}(X;R)$ by
 $$ d(\sigma) \coloneqq \sum_{k = 0}^{n+1} (-1)^k \; \sigma \circ \lambda^k$$
 It is a linear map of degree $-1$, and a calculation shows that $d^2 = 0$.
 \end{Example}

\begin{Definition}[Differential graded modules]

For a graded module $M$, we call a linear map $d \colon M \to M$ of degree $+1$ that satisfies $ d^2 = 0$ a \emph{differential}.
In this case, we call the pair $(M , d)$ a \emph{differential graded module}, or a \emph{complex}.
A morphism $f \colon M \to N$ of differential graded modules $(M,d)$ and $(N,d')$ is a linear map of graded modules that commutes with the 
differentials. That is, the diagram

\centerline{
\xymatrix{
\dotsc \ar[r]^{d} & M^{i-1} \ar[d]^{f^{i-1}} \ar[r]^d & M^i \ar[d]^{f^i} \ar[r]^d &M^{i+1}\ar[d]^{f^{i+1}} \ar[r]^{d} & \dotsc\\
\dotsc \ar[r]^{d'} & N^{i-1} \ar[r]^{d'} & N^{i} \ar[r]^{d'} & N^{i+1} \ar[r]^{d'} & \dotsc}
}
commutes.
\end{Definition}

Often, we shall be interested in complexes $M$ with $M^i = 0$ for all $i < 0$. We call such complexes \emph{cochain complexes}.

\begin{Example}[Singular cochain complex]
 Dualising the singular chain complex of a topological space $X$ yields its \emph{singular $R$-cochain complex}. It is 
 defined as follows: 
 
 $C^n_{sing}(X;R) \coloneqq hom_R(C_n^{sing}(X;R),R)$ and
 $C^*_{sing}(X;R) \coloneqq { \lbrace C^n_{sing}(X;R) \rbrace }_{n \geq 0}$. 
 
 For a moment, denote  the map $d \colon C_{n+1}^{sing}(X) \to C_n^{sing}(X)$ by $\bar{d}$. Define the map
 $d \colon C_{sing}^n(X) \to C_{sing}^{n+1}(X)$ by $\phi \mapsto \phi \circ \bar{d}$. Then ${\bar{d}}^2 = 0$ implies $d^2 = 0$.
 Hence, $d$ is a differential and $(C^*_{sing}(X),d)$ a cochain complex.
\end{Example}

\begin{Definition}
The \emph{homology} of a differential graded module $(M,d)$ is defined as $H(M) \coloneqq 
H(M,d) \coloneqq ker \, d / im \, d$.
We call elements of $ker \, d$ \emph{cocycles} and elements of $im \, d$ \emph{coboundaries}.
Moreover, we call a cochain complex $(M,d)$ \emph{connected} if $H^0(M) \cong R$, and $n$-connected if 
it is connected and $H^i(M) = 0$ for $i = 1, \dotsc, n$. For $1$-connected, we shall also use the term \emph{simply connected}.
\end{Definition}

A morphism $f \colon M \to N$ of differential graded modules commutes with the differentials and hence maps cocycles to cocycles
and coboundaries to coboundaries. Therefore, it induces a well-defined map $H(f) \colon H(M) \to H(N) $ with $[x] \mapsto [f(x)]$. 

\begin{Remark}
 In most other subfields of topology and algebra the above definition would be called \emph{cohomology} since the
 differential raises the degree. However, in rational homotopy theory the term homology seems to be standard for the
 above definition and we decided to stick with this.
\end{Remark}

\begin{Example}[Singular cohomology]
 Given a topological space $X$, the homology $H(C^*_{sing}(X;R))$ is called \emph{singular cohomology} of $X$
 and will be denoted by $H^*(X;R)$. We shall use the convention $H^*(X) \coloneqq H^*(X; \Q)$.	
\end{Example}


Most of the time, we shall mainly be interested in the homology of a chain complex. Therefore, it will be quite useful to replace
a cochain complex by a somehow simpler cochain complex, which has the same homology. This motivates the following definition.

\begin{Definition}
 A morphism $f$ of differential graded modules that induces isomorphisms $H^i(f)$ for all $i \in \Z \;$ is called a
 \emph{quasi-isomorphism}. It will be denoted by $\overset{\simeq}{\longrightarrow}$.
 
 We call two differential graded modules $(M,d)$ and $(N,d)$ \emph{weakly equivalent} if there is a finite chain of 
 quasi-isomorphisms
 $$ (M,d) \overset{\simeq}{\leftarrow} (M_1,d) \overset{\simeq}{\rightarrow} \cdots 
 \overset{\simeq}{\leftarrow} (M_n,d) \overset{\simeq}{\rightarrow} (N,d)$$
\end{Definition}

\begin{Definition}
 Given two graded modules $V$ and $W$ we define their \emph{tensor product}
 $(V \otimes W) = \bigcup_{n \in \Z} (V \otimes W)^n $ by 
 $$ (V \otimes W)^n \coloneqq \bigoplus_{p + q = n} V^p \otimes W^q$$
 The convention $|v \otimes w| \coloneqq |v| + |w|$ for homogeneous elements $v \in V$ and $w \in W$
 makes it a graded module. 
 
 For linear maps $f \colon V \to V'$ of degree p and $g \colon W \to W'$ of degree q, we define the induced linear map
  ${f \otimes g \colon V \otimes W \to V' \otimes W'}$ by
 $$ (f \otimes g) ( x \otimes y) = (-1)^{deg \, x \; deg \,y} f(x) \otimes g(y) $$
  It has degree $p+q$.
 \end{Definition}

Now that we have introduced graded modules, we want to enrich their structure with a multiplication and get to 
the notion of graded algebras.
 
 \begin{Definition}

 A \emph{multiplication} on a graded module M is an associative linear map
 $${m \colon M \otimes M \to M }$$ that has a neutral element $1_M \in M^0$. That is, $m(x \otimes 1_M) = m(1_M \otimes x) = x$.
 In the following, we shall just write $xy \coloneqq m(x \otimes y)$.
 \end{Definition}

 Note that the definition implies $|xy| = |x| + |y|$ for homogeneous elements $x,y \in M$.
 
\begin{Definition}
 A graded module $A$ with a multiplication is called a \emph{graded algebra}. We say that $A$ is \emph{(graded) commutative} if
 $xy = (-1)^{deg \, x \; deg \, y} yx$ for all $x,y \in A$ holds. Furthermore, a \emph{morphism of graded algebras} 
 $f \colon A \to B$ is a morphism of graded modules that respects multiplication. That is, 

 \centerline{$f(xy) = f(x)f(y)$ for all $x,y \in A$ and $f(1_A) = 1_B$.} 
 
\end{Definition}

One should not get confused by the term commutative. It only means commutative in the regular sense in even degrees but
anti commutative if both elements are in odd degrees.

\begin{Example}
 A polynomial ring $R[X]$ with grading $|X^n| = n i$ for a fixed $i \in \N$ becomes a graded algebra by the usual multiplication.
 However, $R[X]$ is not a graded algebra with the grading $|X^n| = 1$ from Example \ref{ex:PolynomialRingGradedModule} and the usual
 multiplication. Here the grading violates the rule that multiplication adds degrees.
\end{Example}


We can also introduce differentials to the world of graded algebras:
\begin{Definition}
 Let $A$ be a (commutative) graded algebra and $d$ a differential on $A$. Further, let $d$ be a
 a \emph{derivation}, which means that it satisfies the Leibniz rule 
 $$ d(xy) = (dx)y + (-1)^{deg x} x(dy).$$
 
 Then, we call $(A,d)$ a \emph{(commutative) differential graded algebra}. If $A^i = 0$ for all $i < 0$, we
 speak of a \emph{cochain algebra}. 
 
 Given two differential graded algebras $(A,d_1)$ and $(B, d_2)$ we can make their tensor product a 
 differential graded algebra $(A \otimes B, d)$ by defining 
 $$ d(a \otimes b) \coloneqq d_1(a) \otimes b + (-1)^{deg \, a} a \otimes d_2(b)$$
\end{Definition}

Note that the homology of a (commutative) graded differential algebra is a (commutative) graded algebra. 

Next, we want to introduce the tensor algebra $V \overset{i}{\to} TV$ of a graded module $V$. It has the nice property
that for every graded module morphism $\varphi \colon V \to A$ to a graded algebra $A$ there is a unique 
 morphism of graded algebras $\psi \colon TV \to A$ that extends $\varphi$. That is, the following diagram commutes:

\centerline{ \xymatrix{
V \ar[dr]_{\varphi} \ar[r]^i & TV \ar@{-->}[d]^{\exists! \psi} \\
& A
}
}

\begin{Definition}
 The \emph{tensor algebra} of a graded $R$-module $V$ is given by
 $$ TV \coloneqq \bigoplus_{j = 0}^\infty T^j V$$
 where $T^0 V \coloneqq R$ and $T^j V \coloneqq \bigotimes_{k=1}^j V$ with the embedding 
 $i \colon V \overset{\cong}{\to} T^1 V$. 
 
 The tensor algebra is a graded module with grading given by
 $ |v| = \sum_{ i = 1}^k |v_i|$ for $ v = \otimes_{i = 1}^k v_i $ (with $v_i \in V$ homogeneous).
 For such $v$ we will say that $v$ has \emph{wordlength} $k$.
 Furthermore, the tensor algebra is a graded algebra with multiplication $xy \coloneqq x \otimes y$.
\end{Definition}

Although the tensor algebra is a very nice algebra it has the problem that it is not commutative in general. We can fix this by
defining the free commutative algebra for rings $R$ where $0 \neq 2$.

\begin{Definition}
 Let $V$ be a free graded $R$-module with $0 \neq 2$ in R. Let $I \subset TV$ be the ideal generated by the elements
 $$ v \otimes w - (-1)^{deg \, v \; deg \, w} w \otimes v$$
 for all $v,w \in V$. Then the quotient algebra $ \Lambda V \coloneqq TV/I$ is called the \emph{free commutative graded algebra} over $V$.
\end{Definition}

\begin{Remark}
 

The free commutative graded algebra is always commutative since we divide out all elements which would prevent it 
from being commutative. It has a similar universal property as the tensor algebra: \\ For every graded
module $V$, commutative graded algebra $A$ and linear morphism $\varphi \colon V \to A$ there is a unique morphism of
graded algebras $\psi \colon \Lambda V \to A$ that makes the following diagram commutative

 

\centerline{\xymatrix{
V \ar[dr]_{\varphi} \ar[r] & \Lambda V \ar@{-->}[d]^{\exists! \psi} \\
& A
}
}
\end{Remark}

To be able to speak about the structure of free commutative algebras, we have to introduce some more notation.

\begin{Definition}
For a free graded module $V$, we denote 
by $\Lambda^{\geq k} V$ the vector space of elements in $\Lambda V$ with wordlength at least $k$.
Additionally $V^{\geq k}$ denotes the subspace (of $V$) of elements with degree at least k and ${\Lambda V}^{\geq k}$ 
is the free commutative algebra over this subspace. We denote by $\Lambda V^{even}$ the free commutative algebra over the elements in $V$ of even degree
and likewise for $\Lambda V^{odd}$.
 
\end{Definition}

Unfortunately, the notations $\Lambda^{\geq k} V$ and ${\Lambda V}^{\geq k}$  look 
quite similar and are different in their nature, one should not confuse them.


Later it will be important to have a good understanding of free commutative algebras and therefore
we shall now provide some examples.

\begin{Example}
\label{ex:FreeCommutativeEvenDegrees}
 To start off simple, consider $\Lambda \langle v \rangle$ with $|v| = 2$. 
 It is isomorphic to the polynomial algebra $R[v]$ (with $|v| = 2$).
\end{Example}

\begin{Example}
\label{ex:FreeCommutativeOddDegrees}
 A similiar example is $\Lambda \langle v \rangle$ with $|v| = 3$. Since 
 $$0 = v^2 - (-1)^{deg \, v \; deg \, v} v^2 = 2 v^2$$
 and $2 \neq 0$ we get that $v^2 = 0$. This shows that all elements in $\Lambda \langle v \rangle$
 have the form $\lambda_1 v + \lambda_2$ for $\lambda_i \in R$ and multiplication is determined by $v^2 = 0$.
\end{Example}

Combining these two behaviours in odd and even degrees we can look at one last example which
illustrates the general behaviour of the free commutative algebra.

\begin{Example}
 Consider the algebra $\Lambda \langle a,b,c \rangle$ with $|a| = 2$ and $|b| = |c| = 3$.
 Analogous to the last example we see that $b^2 = c^2 = 0$. Furthermore,
 $$ bc - (-1)^{deg \, b \; deg \, c} cb = bc + cb = 0$$
 which is equivalent to $bc = - cb$. This determines the complete multiplicative structure 
 and all elements are of the form 
 $$ \sum_{i,j,k \geq 0} \lambda_{i,j,k} \, a^{p_i} b^{q_j} c^{r_k}$$
 with  $q_j, r_k \in \lbrace 0,1 \rbrace$, $p_i \geq 0$ and $\lambda_{i,j,k} \in R$.
\end{Example}

\begin{Remark}
\label{rem:FreeCommutativeSplits}
From the definition we also get an isomorphism 

$$ \Lambda ( V \oplus W) \cong \Lambda V \otimes \Lambda W $$

\end{Remark}
