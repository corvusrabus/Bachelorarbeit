 \section{NP-complete problems concerning Sullivan algebras}
 
 In this section we want to show the 
 $\NPcomplexity$ completeness of certain problems concerning Sullivan algebras. Later we will see that these problems
 translate directly to topological problems. The Theorems and proofs in this section mainly follow  \cite{Lechuga2000}.
 
 In the following we want our ground ring $R$ to be  $\Q$, hence we only consider rational Sullivan algebras.
 
 \begin{Definition}
  A Sullivan algebra $(\Lambda V, d)$ is called \emph{elliptic} if both $\pi^*(\Lambda V,d)$ and $H(\Lambda V,d)$ have
  finite dimension.
 \end{Definition}
 
 The first result we want to establish is the following:
 
 \begin{Theorem}[Lechuga, Murrilo]
 \label{cohomologyFinTheorem}
  It is a $\NPcomplexity$-hard problem to decide if a given a Sullivan algebra $(\Lambda V,d)$ with $\pi^*(\Lambda V,d)$ finite dimensional 
  is elliptic . 
 \end{Theorem}
 
 
 We will prove this Theorem by reducing $k$-COLOUR to deciding if $(\Lambda V,d)$ is elliptic for some special 
 Sullivan algebra $(\Lambda V,d)$ which we will construct as follows : \\
 
 \begin{Construction}
 \label{constructionOfSullivanAlgebra}
 Let $G = (V,E)$ be an undirected, simple, connected, finite graph with vertices $ V = \lbrace v_1, \dotsc , v_n \rbrace $
 and edges $ E = \lbrace (v_r, v_s) \; | \; (r,s) \in J \rbrace$. From this we construct for a given $k \in \N$ the following
 Sullivan algebra $(\Lambda V_{G,k} , d)$ : \\
 
 $ V^{even}_{G,k} \coloneqq \langle x_1, \dotsc , x_n \rangle $ \; with \; $|x_i| = 2$ \; for \; $ i = 1, \dotsc , n$ \; 
 and \; $dx_i \coloneqq 0$ \\
 
 $V^{odd}_{G,k} \coloneqq \langle y_{(r,s)} \rangle$ , $(r,s) \in J$ \; with \; $|y_{(r,s)}| = 2k - 3$ \; and \; $dy_{(r,s)} \coloneqq 
 \sum_{l = 1}^k x_r^{k -l} x_s^{l - 1}$ \\
 
 \end{Construction}

  If $k \geq 3$ the differential of the $y_{(r,s)}$ contains no linear term, hence
  $(\Lambda V_{G,s} ,d)$ is minimal for $k \geq 3$. To prove Theorem \ref{cohomologyFinTheorem}, we first need the following Lemmata:
 
 \begin{Lemma}
  Let $\K$ be an algebraically closed field and $(\Lambda V,d)$ a minimal Sullivan
  algebra over $\K$ with finite dimensional $V$. \newline5
  Then $H^*(\Sullivan)$ has finite dimension if and only if   
  the only morphism of differential graded algebras 
  $$ \varphi : \Sullivan \to (\K[\alpha],0) \; |\alpha| = 2 $$ 
  is the trivial one.
 \end{Lemma}
 For one direction we don't need algebraically closed, thus we first show :
  \begin{Lemma}
   Let $(\Lambda V,d)$ be a minimal Sullivan algebra with finite $V^i$ for all $i \geq 0$. \newline
  If $H^*(\Sullivan)$ has finite dimension then
  the only morphism of differential graded algebras 
  $$ \varphi : \Sullivan \to (\K[\alpha],0) \; |\alpha| = 2 $$ 
  is the trivial one.
  \end{Lemma}

 \begin{proof}
 Given a morphism  $ \varphi : \Sullivan \to (\K[\alpha],0) \; |\alpha| = 2 $, we want to inductively show that
 $\varphi$ is zero on elements in $V^{\leq i}$. So if we have that $\varphi$ is zero on $V^{< i}$ we look at the
 subalgebra generated by elements above degree $i$ namely $(\Lambda V^{\geq i} , \bar{d})$ with the induced differential $\bar{d}$.
 %TODO hier noch genauer mit den Zyklen werden
 $\Sullivan$ being a minimal Sullivan algebra implies that every $ v \in V^i$ is a $\bar{d}$-cocycle. If we look at powers
 $[v^k]$ of $[v]$ the finiteness condition on homology implies that this will land in a $0$-module for big enough $k$, hence 
 $[z^k] = 0$. This is equivalent to $v^k$ being a coboundary. Now consider the induced morphism
 $$ \bar{\varphi} \colon (\Lambda V^{\geq i} , \bar{d}) \to (\K[\alpha],0)$$
 Since boundaries get mapped to boundaries $\bar{\varphi}(v^k)$ is a coboundary. Besides, $(\K[\alpha],0)$ is zero in odd degrees thus
 $0$ is the only coboundary and therefore $\varphi(v^k) = 0$. Now that $(\K[\alpha],0)$ is integer we have 
 $\varphi(v)= \bar{\varphi}(v) = 0$ and this closes the induction.
   
 \end{proof}

 
\begin{Lemma}
\label{lma:cohomoly+equations}
 $H^*(\Lambda V_{G,k})$ has infinite dimension $\iff$ The system of equations \\
 \begin{equation}
 \label{systemofequations}
 {\lbrace \sum_{l = 1}^k u_r^{k - l} u_s^{l - 1} = 0 \; | \; (r,s) \in J \rbrace}  
 \end{equation}
 
 has a non trivial solution 
 $(\lambda_1 , \dotsc, \lambda_n) \in \C^n$
\end{Lemma}

\begin{proof}
 We use the result of Halperin (\cite{Halperin1988} p. 6) that $H^*(\Lambda V_{G,k})$ has infinite dimension if and only if %TODO Referenz einfügen!
 there is a non trivial morphism of differential graded algebras \\
 ${\varphi \colon (\Lambda V_{G,k},d) \to ( \C [\alpha] ,\acute{d} = 0)}$ with $|\alpha| = 2$. How can such a morphism look like?
 Clearly $\varphi(x_i) = \lambda_i \alpha$ for some $\lambda_i \in \C$ and $\varphi(y_{(r,s)}) = 0$ for all $(r,s) \in J$ , since 
 $(\C [\alpha] , 0)$ is zero in odd degrees. Furthermore, $\varphi$ must commute with the differentials, hence 
 for all $(r,s) \in J$
 $$ 0 = \acute{d} (\varphi(y_{(r,s)})) = \varphi(dy_{(r,s)}) = \varphi(\sum_{l = 1}^k x_r^{k -l} x_s^{l - 1})
 = \sum_{l = 1}^k \varphi(x_r)^{k -l} \varphi(x_s)^{l - 1}$$
 
 This shows that $(\varphi(x_i))_{i = 1, \dotsc , n}$ must be a solution of \ref{systemofequations}, which is not trivial
 if $\varphi$ is not trivial.
 This also works the other way round and every non trivial solution  $(\lambda_1 , \dotsc, \lambda_n)$ of \ref{systemofequations}
 can be used to define a non trivial morphism $\varphi(x_i) = \lambda_i \alpha$.
 \end{proof}

 \begin{Theorem}
   $G = (V,E)$ is k-colourable $\iff$ $(\Lambda V_{(G,k)},d)$ is not elliptic
 \end{Theorem}

 \begin{proof}
  From the construction of $(\Lambda V_{(G,k)},d)$ we see that  $\pi^*(\Lambda V_{(G,k)},d)$ is finite dimensional, therefore
  we only have to care about its cohomology. Suppose now that $G$ is $k$-colourable and we have a colouring
  $f \colon V \to { \lbrace 1, \dotsc , k \rbrace }$ with $f(x_r) \neq f(x_s)$ for $(r,s) \in J$. Let $\zeta_k$ be a primitive 
  $k$-th root of unity, then for $(r,s) \in J$ it holds that
  
  $$ \sum_{l = 1}^k (\zeta_k^{f(x_r)})^{k-l} (\zeta_k^{f(x_s)})^{l-1}
  = \frac{(\zeta_k^{f(x_r)})^{k} - (\zeta_k^{f(x_s)})^{k}}{ \zeta_k^{f(x_r)} - \zeta_k^{f(x_s)}} = 0
  $$
  
  Hence $(\zeta_k^{f(x_i)})_{i = 1, \dotsc, n}$ defines a non trivial solution of \ref{systemofequations}
  and Lemma \ref{lma:cohomoly+equations} tells us that $(\Lambda V_{(G,k)},d)$ is not elliptic and this shows
  ``$\implies$''. \\
  Let now $(\Lambda V_{(G,k)},d)$ be not elliptic. Then Lemma \ref{lma:cohomoly+equations} gives us a non trivial
  solution $(\lambda_1 , \dotsc, \lambda_n) \in \C^n$ of \ref{systemofequations}. We now use this solution to construct
  a colouring of $G$. First observe that $G$ being connected implies that $\lambda_i \neq 0$ for all $\lambda_i$. Then we have
  for $(r,s) \in J$ that $ \lambda_r^k - \lambda_s^k = ( \lambda_r - \lambda_s) 
  \sum_{l = 1}^k \lambda_r^{k - l} \lambda_s^{l - 1} = 0$ . Since $G$ is connected we have 
  $\lambda_1^k = \dotsc = \lambda_n^k$ , which we can wlog assume to equal $1$. Therefore every $\lambda_i$ is a 
  $k$-th root of unity and we can define $f \colon V \to { \lbrace 1, \dotsc , k \rbrace }$ such that 
  $\lambda_i = \zeta_k^{f(x_i)}$. This $f$ now defines a colouring, because if we assume that for a $(i,j) \in J$
  $f(x_i) = f(x_j)$ holds we also get that 
  $$\sum_{l = 1}^k \lambda_i^{k - l} \lambda_j^{l - 1} = \sum_{l = 1}^k (\zeta_k^{f(x_i)})^{k-l} (\zeta_k^{f(x_j)})^{l-1}
  = \sum_{l = 1}^k (\zeta_k^{f(x_i)})^{k-1} = k (\zeta_k^{f(x_i)})^{k-1} \neq 0$$ 
  which is a contradiction to $(\lambda_1 , \dotsc, \lambda_n)$ being a solution of \ref{systemofequations}.
  
 \end{proof}

 Since the construction \ref{constructionOfSullivanAlgebra} is polynomial this also proves Theorem \ref{cohomologyFinTheorem}.
