\section{Sullivan algebras}

In this section we introduce Sullivan algebras which form our most important algebraic tool. The theory
developed in  Section~\ref{sec:FromSpacesToAlgebras} will allow us to assign a (up to isomorphism) unique Sullivan algebra
to every topological space. In this case, we shall also speak of a model of the space. For a large class of spaces,
the model will encode the rational 
homotopy type, which will be introduced later, of the accompanying space.
% 
% One of the nicest things about rational homotopy theory is that we can assign to every space a (up to a
% suited isomorphism) unique algebraic model in form of a so called \emph{Sullivan algebra}. This theory
% will be sketched in section \ref{sec:FromSpacesToAlgebras}. The models will contain a lot of information
% about the accompanying space (what this means will be formalised in section \ref{sec:DefinitionsOfTopologicalInvariants})
% and they also have the property that some subclasses of them can be
% represented in a computer which will allow us to make statements about the computational
% complexity of certain topological problems.


\subsection{Definition and examples of Sullivan algebras}
Although most definitions and Theorems would work fine for a general ground ring, we fix for 
all modules and algebras considered in this section the ground ring $\Q$. Therefore, we shall mainly speak of 
graded vector spaces instead of graded modules.

\begin{Definition}
 Given a free graded vector space $V = {\lbrace {V^i}\rbrace}_{ i \geq 1} $ we call $(\Lambda V, d)$ a \emph{Sullivan algebra} 
 if it satisfies the following so called \emph{nilpotence} property:
 
  There exists a filtration of graded subspaces $V(0) \subseteq V(1) \subseteq V(2) \subseteq \cdots \subseteq V$
  with $ V = \bigcup_{k = 0}^{\infty} V(k)$ such that 
  $$ d(V(0)) = 0 \; \text{and} \; d( V(k)) \subseteq  \Lambda V(k-1) $$
  
 Furthermore, a Sullivan Algebra $(\Lambda V,d)$ is called \emph{minimal} if $im(d) \subseteq \Lambda^{\geq 2} V$
\end{Definition}

Note that the structure of a Sullivan algebra $\Sullivan$ is completely determined by  $V$ and its
differential $\restr{d}{V}$. That is, if we are given a linear map $d \colon V \to \Lambda V$ the
universal property of free commutative algebras tells us that $d$ extends uniquely to a derivation $d \colon \Lambda V \to \Lambda V$.
Therefore, it is enough to specify the differential of a Sullivan algebra $\Sullivan$ on a basis of $V$ (such that
it will satisfy $d^2 = 0$).


\begin{Proposition}
\label{prop:WellBehavedFiltrations}
 Let $\Sullivan$ be a differential graded algebra with $V = {\lbrace V^n \rbrace}_{n \geq 2}$ and 
 $im \;d \subseteq \Lambda^{\geq 2} V$. Then $\Sullivan$ is a minimal Sullivan algebra.
 The filtration can be chosen such that $V(0) = V^2$ and
 $V(i) = V^{i+2} \oplus V(i-1)$ for $i \geq 1$.
\end{Proposition}
\begin{proof}
 The proof is mainly a degree argument. We choose the filtration $V(i) = V^{i+2} \oplus V(i-1)$ from the claim
 and show that it satisfies
 the nilpotence condition. Begin with $V(0) = V^2$ and take $v \in V(0)$ then $dv$ lies in
	degree $3$ and $dv \in \Lambda^{\geq 2} V$. Since $V^0 = V^1 = 0$, it follows that ${(\Lambda V)}^3$ only contains 
 nonzero elements of wordlength 1, therefore $dv = 0$. 
 It remains to show that  $d(V^{k+2}) \subseteq \Lambda V^{ \leq k+1}$ holds for $k \geq 1$. Take $v \in V^{ k+2}$ then 
 $| dv | = k + 3$ and basically the same argument as before works; if $dv$ contained a term $xy \notin \Lambda V^{\leq k+1}$
 then $|x| = 1$ and $|y| = k + 2$ or $|x| = 0$ and $|y| = k + 3$ but in both cases we have $x = 0$ thus $xy = 0$.
\end{proof}

\begin{Example}[A Sullivan algebra which is not minimal]
\label{ex:AlgebraConstructedFromK3}
 Consider the graded $\Q$-vector space $V \coloneqq \langle a,b,c, x,y,z \rangle$ with $|a| = |b| = |c| = 2$,
 $|x| = |y| = |z| = 1$. Define a morphism $d \colon V \to \Lambda V$ by 
 \begin{align*}
 da = db = dc = 0 & &
 dx = a + b & & dy = b + c & & dz = c + a.  
 \end{align*}

 
 Let us exemplary check on one generator 
 that $d$ induces a differential. We have 
 $$d^2(x) = d(a + b) = da + db = 0$$ and the other generators work similar.
 Further, we choose the filtration $V(0) = \langle a,b,c \rangle$, $V(i) = V$ for $i \geq 1$ and it is 
 immediate that it satisfies the nilpotence condition. Hence, $\Sullivan$ is a Sullivan algebra. Moreover,
 $\Sullivan$ is not minimal since $dx = a + b$ contains the elements $a$ and $b$ of wordlength $1$.
\end{Example}


\begin{Example}[An algebra which is not a Sullivan algebra]
 Consider  $V \coloneqq \langle x,y,z \rangle$ with $|x| = |y| = |z| = 1$ and 
 equip it with the differential $d \colon V \to \Lambda V$  defined by $dx = yz$, $dy = xz$, $dz = xy$.
 Note that the differential is well-defined as can be seen by a calculation.
 
 We show that $\Sullivan$ is not a Sullivan algebra.
 For this, suppose that $\Sullivan$ defines a Sullivan algebra. This yields a filtration ${\lbrace V(i) \rbrace}_{i \geq 0}$ satisfying
 the nilpotence condition. Let $i$ and $j$ be the least integers such that $x \in V(i)$ and $y \in V(j)$. We get
 $i < j$, since $dx = yz \in \Lambda \langle y,z \rangle$ and $j < i$, since $dy = xz \in \Lambda \langle x,z \rangle$.
 This contradicts the existence of the filtration, hence $\Sullivan$ is not a Sullivan algebra. \\
 Note that we indeed need $3$ generators for such an example since all differential graded algebras of the form
	$\Sullivan$ with $ V = {\lbrace V^i \rbrace}_{i \geq 1}$ and $dim (V) \leq 2$ are Sullivan algebras. This can be 
 seen by checking all cases of possible differentials on $\Sullivan$.
 \end{Example}
 
 Later, we shall be interested if the homology of a Sullivan algebra is finite dimensional. A simple criterion 
 for this is the following:
 \begin{Lemma}
  Let $(\Lambda V,d)$ be a free commutative algebra with differential and $V = {\lbrace V_i \rbrace}_{i \geq 1}$ finite dimensional.
  The following are equivalent:
  
  \begin{enumerate}
   \item\label{itm:lemmaItemOne} $(\Lambda V,d)$ has finite dimensional homology, i.e. $H^i \Sullivan$ is finite dimensional for every
   $i$ and there is $n_0$ such that  $H^n \Sullivan = 0$ for $n \geq n_0$.
   \item\label{itm:lemmaItemTwo} For a basis $ v_1, \ldots, v_n$ of $V^{even}$ there is 
  $i \in \N$ with $[v^i_j] = 0$.
  \end{enumerate}

 \end{Lemma}
 
 \begin{proof}
  Choose $n_0$ as in the claim we then have that $v^{n_0} \in H^{|v| \, n_0} \Sullivan = 0$ for all $v \in V$ and hence 
  $\ref{itm:lemmaItemOne}$ implies $\ref{itm:lemmaItemTwo}$.
  Let now $v_1, \ldots, v_n$ be a basis of $V^{even}$ and $i \in \N$ such that $v_j^i = 0$ for all $j = 1, \ldots, n$. Let
  $w_1, \ldots, w_m$ be a basis of $V^{odd}$. We know that $\Sullivan$ is generated by finite products
  of $v_1, \ldots, v_n, w_1, \ldots, w_m$  hence  $H \Sullivan $ is generated by finite 
  products of $[v_1], \ldots, [v_n], [w_1], \ldots, [w_m]$ (since $H \Sullivan $ inherits the multiplicative structure from
  $\Sullivan$). Since $[w_j^2] = 0$ (the $w_j$ lie in odd degrees) and $[v_j^i] = 0$, there are only finitely
  many nonzero products that can be formed like this and hence $H \Sullivan$ is finite dimensional.
 \end{proof}

 \begin{Example}
\label{ex:AlgebraConstructedFromK3HasFiniteHomology}
 We can see that the Sullivan algebra $\Sullivan$ from~\ref{ex:AlgebraConstructedFromK3} has finite dimensional homology
 as follows. 
 We have 
 $$d \big{(} \frac{x - y + z}{2} \big{)} = \frac{(a + b) - (b + c) + (c + a)}{2} = a \in im \, d$$
 and hence $[a] = 0$. A similar calculation shows that $b,c \in im \, d$ and therefore
 $[a] = [b] = [c] = 0$. Hence, by the last Lemma $H \Sullivan$ is finite dimensional.
 \end{Example}

 Actually, the Sullivan algebra from last example is not far away from having infinite dimensional homology.
 We can see this as follows:
 
 \begin{Example}
\label{ex:AlgebraFromK3WithoutOneEdge}
  Consider the Sullivan algebra $\Sullivan$ given by $V = \langle a,b,c , x,y \rangle$ with 
  $|a| = |b| = |c| = 2$, $|x| = |y| = 1$, $da = db = dc = 0$, $dx = a + b$ and $dy = b + c$.
  Note that this is the Sullivan algebra from last example without the generator $z$.
	 We have that ${(\ker \, d)}^2 = \langle [a], [b], [c] \rangle$ and 
  $(im \, d)^2 = \langle (a + b), (b +c) \rangle$. Hence $dim \, ( (ker \, d)^2 ) = dim \, ((im \, d)^2 )+ 1 = 3$ and 
  we can easily see that $c \notin im \,d$, i.e. $[c] \neq 0$. Further, all elements in $im \, d$ are 
  linear combinations of products of $( a + b)$ and $(b +c)$. This shows that $[c^i] \neq 0$ for all $i \in \N$
  and therefore $H \Sullivan$ has infinite dimension. 
 \end{Example}

\begin{Remark}
 The way we defined Sullivan algebras so far was toilsome and often long. Now that the reader has become more familiar
 with Sullivan algebras we introduce a short notation to describe them. \\
 We write $\Lambda (a_i , b_j, \ldots, x_k ; d x = a)$ to mean the Sullivan algebra
 $(\Lambda \langle a, b, \ldots, x \rangle , d) $ with $|a| = i$, $|b| = j$, $|x| = k$, $d x = a$ and
 $dy = 0 $ for all other generators $y$. The subscripts denote the degree of the indexed generator and
 we only specify the differential on generators which get not mapped to zero. \\
  Note that the special case $\Lambda(a,b, \ldots ;)$ describes a Sullivan algebra with zero differential.
\end{Remark}


Next, we want to see that Sullivan algebras provide ``good understandable models'' for the homology of 
commutative cochain algebras. For this we need:

\begin{Definition}
  A \emph{Sullivan model} of a commutative cochain algebra $(A,d)$ is a Sullivan algebra $\Sullivan$ with a quasi-isomorphism
  $\varphi \colon \Sullivan \overset{\simeq}{\to} (A,d)$. If $\Sullivan$ is minimal we speak of a 
  \emph{minimal Sullivan model}.
\end{Definition}
Sometimes we shall leave out the quasi-isomorphism and call the algebra $\Sullivan$ a Sullivan model.

\begin{Example}
\label{ex:MinimalModelOfSpheres}
 To become familiar with this concept we compute a minimal Sullivan model of the cohomology 
 algebra $H^*(S^n) \cong (\Q [\alpha] / (\alpha^2), 0)$ (with $|\alpha| = n$) of the
 $n$-sphere. If $n$ is odd this is fairly easy since we can define 
 \begin{align*}
  \varphi \colon \Lambda (v_n;) \to H^*(S^n) & & v \mapsto \alpha
 \end{align*}
 This defines a quasi-isomorphism since $n$ being odd implies
 $ v^2 = 0$. \\
 For $n$ even there is the problem that $v^2 \neq 0$. Therefore, also 
 $[v^2] \neq 0$ and for the same $\varphi$ as before we would get
 $(H^{2n}(\varphi)) ([v^2]) = 0$ and $\varphi$ would not be quasi-isomorphism since $H(\varphi)$ is not injective.
 Thus, we have to ``kill'' $v^2$
 in homology by introducing a new element that maps to $v^2$. For this we define
  \begin{align*}
  \psi \colon \Lambda(v_n, x_{2n -1}; dx = v^2) \to H^*(S^n)& & \text{by}  &&v \mapsto \alpha& &\text{and}& & x \mapsto 0 
 \end{align*}
 
 This defines a quasi-isomorphism. Since both constructed algebras are minimal, they are minimal models
 of the corresponding cohomology algebras.
\end{Example}

\begin{Remark}
\label{rem:MinimalSullivanModelsExist}
Every connected commutative cochain algebra has a minimal Sullivan model
(\cite{Felix2001} Theorem 14.12). However, the proof of this is not constructive. We shall prove a weaker 
version of this fact, which can be obtained algorithmically and will suffice for later use.
\end{Remark}

\begin{Theorem}
\label{thm:MinimalSullivanModelsExistForSimplyConnected}
 Every simply connected, commutative cochain algebra has a minimal Sullivan model.
\end{Theorem}

In the case of the last theorem, the model can be obtained by an algorithm, which we will use to prove the theorem.

\begin{Algorithm}
\label{alg:ConstructionOfMinimalSullivanAlgebra}
 Let $(A,d)$ be a simply connected cochain algebra. We inductively construct a Sullivan algebra $\Sullivan$ and a morphism
 $\varphi \colon \Sullivan \to (A,d)$ as follows:
 
 \begin{enumerate}
	 \item\label{itm:FirstAlgorithmStep} Choose a basis ${([v_i])}_{i \in I}$ of $H^2(A,d)$ and let 
	 ${(v_i)}_{i \in I} \in A^2$ be representing cocycles.
	  Define $V^2 \coloneqq \langle {(\alpha_i)}_{i \in I} \rangle$ and
  $\varphi_2 \colon (\Lambda V^2,0) \to (A,d)$ by $\varphi_2(\alpha_i) = v_i$.
  This induces an isomorphism 
  $H^2(\varphi_2) \colon H^2(\Lambda V^2, 0) \overset{\cong}{\to} H^2(A,d)$.
   
   We now proceed inductively for $k \geq 2$. Assume that we have constructed $(V^{\leq k}, d)$ and
   ${\varphi_k \colon (\Lambda V^{\leq k}, d) \to (A,d)}$ such that $\varphi_k$ is a isomorphism in homology in degrees $\leq k$,
   and injective in homology in degree $k+1$.
   In the following steps, we construct $V^{k+1}$ and 
   extend $\varphi_k$ to a morphism $\varphi_{k+1} \colon (\Lambda V^{\leq k+1}, d) \to (A,d)$ with 
   $ \restr{\varphi_{k+1}}{V^{\leq k}} = \varphi_k$,
   $H^{k+1}(\varphi_{k+1})$ bijective, and $H^{k + 2}(\varphi_{k+1})$ injective:
   
   \item\label{itm:first} Choose elements ${\lbrace v_i \rbrace}_{i \in I} \subset A^{k+1}$ such that
   
   $$H^{k+1}(A,d) = im ( H^{k+1}(\varphi_k)) \; \oplus \; \bigoplus_{i \in I} \; \langle [v_i] \rangle $$
   
   \item\label{itm:third} Further, choose cocycles ${ \lbrace c_j \rbrace}_{j \in J} \subset {(\Lambda V^{\leq k})}^{k+2}$
   such that ${ \lbrace [c_j] \rbrace}_{j \in J}$ forms a basis of $ker ( H^{k+2}(\varphi_k))$.
   
   \item By construction, the ${\lbrace \varphi_k(c_j) \rbrace}_{j \in J}$ are coboundaries. Hence, we can choose 
   ${\lbrace w_j \rbrace}_{j \in J} \subset A^{k+1}$ with $\varphi_k (c_j) = d{w_j}$.
   
   \item\label{itm:second} Define 
   $V^{k+1}$ as the vector space generated by the formal basis $(\alpha_i)_{i \in I} \cup (\beta_j)_{j \in J}$. We extend $d$
   to a differential in $\Lambda V^{\leq k+1}$ by $d \alpha_i = 0$ and $d \beta_j = c_j$. Moreover, we extend $\varphi_k$
   to $\varphi_{k+1} \colon \Lambda V^{ \leq k+1} \to A$ by $\varphi_{k+1} (\alpha_i) = v_i$ and 
   $\varphi_{k+1} (\beta_j) = w_j$. Now go to~\ref{itm:first} and repeat the process for $k+1$.
 \end{enumerate}

\end{Algorithm}
We now collect some properties of this algorithm.
\begin{Proposition}
\label{prop:PropertiesOfAlgorithm}
 Let $(A,d)$ be a simply connected cochain algebra. Let $\varphi \colon \Sullivan \to (A,d)$ be the morphism that
 Algorithm~\ref{alg:ConstructionOfMinimalSullivanAlgebra} constructs for $(A,d)$. Then, the following  holds:
 \begin{enumerate}
  \item $\varphi \colon \Sullivan \to (A,d)$ is a minimal Sullivan model of $(A,d)$.
  \item Let $r$ be the least integer greater zero with $H^r(A) \neq 0$. Then, $V^i = 0$ for $1 \leq i < r$ and
    $H^r(\varphi) \colon V^r \overset{\cong}{\to} H^r(A)$.
  \item Let $r$ be the least integer, or infinity, such that $H^r(A)$ is infinite dimensional.  Then,
    all $V^k$ are finite dimensional for $k < r$.
 \end{enumerate}

\end{Proposition}
Note that this proposition also proves~\ref{thm:MinimalSullivanModelsExistForSimplyConnected}.
\begin{proof}
\begin{enumerate}
 \item 

 First, observe that $\Sullivan$ is a Sullivan algebra. By construction, the filtration $V(i) \coloneqq V^{i+2}$ satisfies
 the nilpotence condition. Further, the choice of the differential in Step~\ref{itm:second} 
 implies that $\Sullivan$ is minimal. 
 
 Next, we want to see that $\varphi$ is a quasi-isomorphism. For this, we show by induction over $k \geq 2$ that
 $H^i(\varphi_k)$ is an isomorphism for $i \leq k$ and injective for $ i = k +1$. The case $k = 2$ is clear since $\varphi_2$ was chosen
 as an homology isomorphism in degree two in Step~\ref{itm:FirstAlgorithmStep}, and ${(\Lambda V^2)}^3 = 0$.
 Thus, suppose we have shown the claim for $k \geq 2$.
 Since $\varphi_{k+1}$ extends $\varphi_{k}$, $H^i(\varphi_{k+1})$ is surjective for $i \leq k$. Further, by the choice
 of elements in Step~\ref{itm:first}, it is surjective for $i = k + 1$.
 By induction hypothesis, $H^i(\varphi_{k + 1})$ is injective  for $i \leq k$.
 Since $V^1 = 0$, all elements $[z] \in ker ( H^{k+2}(\varphi_{k+1}))$ lie in $\Lambda V^{\leq k}$ and therefore are
 coboundaries by Step~\ref{itm:third}. Thus, $H^{k+2}(\varphi_{k+1})$ is injective.
 
 Finally, consider $[z] \in ker ( H^{k+1} (\varphi_{k+1}))$ and choose a representing cocycle $z \in (\Lambda V^{\leq k+1})^{k+1} $.
 We can write, using the notation from
 the algorithm,
 $$z = \sum_{i \in I} \alpha_i \lambda_{\alpha_i} + \sum_{j \in J} \beta_j \lambda_{\beta_j} + r$$
 with $r \in \Lambda V^{ \leq k}$.
 Applying the differential to this equation yields
 $$ -dr = d(\sum_{j \in J} \beta_j \lambda_{\beta_j}) = \sum_{j \in J} \lambda_{\beta_j} c_j.$$
 Taking homology in this new equation results in $ [0] = \sum_{j \in J} \lambda_{\beta_j} [c_j]$.
 By the choice of the $c_j$ we have
 $[c_j] \neq 0$ for all $j \in J$. Thus, $\lambda_{\beta_j} = 0$ for all $j \in J$, and 
 therefore $dr = 0$. We also have
 $$0 = H^{k+1}(\varphi_{k+1})([z]) = H^{k+1}(\varphi_{k+1})( \sum_{i \in I} [\alpha_i] \lambda_{\alpha_i} ) 
 + H^{k+1}(\varphi_k)([r])
    = \sum_{i \in I} \lambda_{\alpha_i} [v_i] + H^{k+1}(\varphi_{k})([r]) .$$   
 By the choice of the $[v_i]$ it follows that $\lambda_{\alpha_i} = 0$. Therefore, 
 $ [z] = [r] \in H^{k+1} (\Lambda V^{\leq k})$ and by induction hypothesis $[z] = 0$ follows.

 \item Follows immediately from the construction.
 \item If $r=2$ this is clear, so let $r > 2$. Then, $V^2 \cong H^2(A,d)$ is finite dimensional. Now we can proceed inductively. 
 By construction, $V^k$ is a
 direct sum of a subspace of $H^k(A,d)$ and a subspace of $(\Lambda V^{ <k })^k$. Therefore, it is finite dimensional since
 the latter two are by induction hypothesis and the requirement for $k<r$.
 \end{enumerate}
\end{proof}

\begin{Theorem}
\label{thm:UniquenessOfModels}
 If $(A,d)$ is a connected commutative cochain algebra, then all minimal Sullivan models of $(A,d)$ are isomorphic.
\end{Theorem}
\begin{proof}
 See~\cite{Felix2001} Theorem 14.12.
\end{proof}


% 
% The next definition is about homotopy groups of a Sullivan algebra, of course they bear that name since we will later
% associate them with homotopy groups of a topological space which gets modeled by the Sullivan model.
% 
% \begin{Definition}
%  Let $\Sullivan$ be a Sullivan algebra, we define its complex of \emph{indecomposables} as
%  $(Q(\Lambda V),\bar{d}) \coloneqq (\Lambda V / \Lambda^{\geq 2} V , \bar{d})$ where $\bar{d}$ is the induced differential.
%  Given this we define the \emph{(dual) homotopy groups} of $\Sullivan$ by 
%  $$ \pi^n ( \Sullivan) \coloneqq H^n(Q(\Lambda V),\bar{d}) \quad \text{for $n \geq 1$}$$
% \end{Definition}
% 
% \begin{Remark}
%  If $\Sullivan$ is a minimal Sullivan algebra then its homotopy groups can be obtained fairly simple. The minimality property
%  implies that the differential on the complex of indecomposables of $\Sullivan$ is zero and hence taking
%  homology does not change anything. Therefore, 
%  $$ \pi^n ( \Sullivan) = H^n(Q(\Lambda V),0) \cong V^n$$
% \end{Remark}

%\subsection{Relative Sullivan algebras}

We have now seen how commutative cochain algebras can be modelled. In the following, we shall generalise the notion of
a Sullivan algebra to be able to model morphisms of commutative cochain algebras.

\begin{Definition}
 We call $(B \otimes \Lambda V,d)$ a \emph{relative Sullivan algebra} if it satisfies the following:
 
 \begin{itemize}
  \item $(B,d) \coloneqq (B \otimes 1, d)$ is a a connected sub cochain algebra.
  \item $1 \otimes V = V = { \lbrace V^p \rbrace}_{ p \geq 1}$ is a graded vector space.
  \item There is a filtration $V(0) \subseteq V(1) \subseteq V(2) \subseteq \ldots$ of graded subspaces
    with  $ V = \bigcup_{k = 0}^{\infty} V(k)$ such that 
    \begin{align*}
     d(V(0)) \subseteq B &  & \text{and} & & d(V(k)) \subseteq B \otimes \Lambda (V(k-1)) \quad \text{for $k \geq 1$.}
    \end{align*}
 \end{itemize}
 Moreover, $(B \otimes \Lambda V,d)$ is called \emph{minimal} if 
 $im \, d \subseteq B^{\geq 1} \otimes \Lambda V + B \otimes \Lambda^{\geq 2} V$.
\end{Definition}

\begin{Remark}
 Note that for $B = \Q$ the definition of Sullivan algebra and relative Sullivan algebra are the same.
\end{Remark}

\begin{Definition}

 A \emph{Sullivan model} of a morphism $\varphi \colon (B,d) \to (C,d)$ of commutative cochain algebras is
 a relative Sullivan algebra $\RelSullivan$ and a quasi-isomorphism
 $$ m \colon \RelSullivan \overset{\simeq}{\to} (C,d) $$
 such that $\restr{m}{B \otimes 1} = \varphi$.
\end{Definition}

\begin{Proposition}
\label{prop:ExistenceMorphismSullivanModels}
 A morphism $\varphi \colon (B,d) \to (C,d)$ of commutative cochain algebras has a Sullivan model
 if $H^0(B) = \Q = H^0(C)$ and $H^1(\varphi)$ is injective.
\end{Proposition}
\begin{proof}
 See~\cite{Felix2001} (Proposition 14.3).
\end{proof}

\subsection{Properties of Sullivan Algebras}
Here we want to introduce some notations for properties of Sullivan algebras. Sometimes it will not be
immediately clear why some properties of Sullivan algebras could be interesting and why we give them a name. We assure
the reader that we shall see a correspondence of these definitions to topological invariants.

\par

Let $\Sullivan$ be a Sullivan algebra. Consider the projection 
$\varrho_m \colon \Sullivan \to (\Lambda V / \Lambda^{> m} V , d)$ for $m \geq 1$, then $H^1(\varrho_m)$ is injective.
Therefore, Proposition~\ref{prop:ExistenceMorphismSullivanModels} tells us that there is a Sullivan model

\centerline{
\xymatrix{
\Sullivan \ar[r]^(.35){\lambda_{m}} \ar[rd]_{\varrho_m} &
(\Lambda V \otimes \Lambda Z(m) , d)
\ar[d]^{\zeta_m}_{\simeq} \\
 & (\Lambda V / \Lambda^{> m} V , d)
}
}

of $\varrho_m$. In the diagram $\Lambda Z(m)$ denotes the Sullivan algebra which models $\varrho_m$ in dependence of $m$.

\begin{Definition}
\label{def:LSCategory}
 The \emph{(Lusternik-Schnirrelmann) category} of a Sullivan algebra $\Sullivan$ is the 
 least integer $m$ (or $\infty$) such that there exists a morphism 
 $\pi_m \colon (\Lambda V \otimes \Lambda Z(m) , d) \to \Sullivan$ with $\pi_m \circ \lambda_m = id_{\Lambda V}$.
 It will be denoted by $cat \, \Sullivan$.
\end{Definition}

\begin{Definition}
 The \emph{product length} of a cochain algebra $A$ is the smallest integer $n$ (or $\infty$) such that 
 $\Pi_{i = 1}^{n+1} a_i = 0$  for all $a_i \in A$. It will be denoted by $nil \,A$.
\end{Definition}

\begin{Example}
 The product length of $\Lambda \langle x \rangle$ is 2 for $|x|$ odd and $\infty$ for $|x|$ even.
\end{Example}

% % \begin{Proposition}
% % %  \begin{enumerate}
% % %   \item 
% %   A Sullivan algebra $\Sullivan$ has category $ \leq m$ if and only if there exists a diagram of commutative cochain
% %   algebras 
% %   
% %   \centerline{
% %   \xymatrix{
% %   \Sullivan \ar[r]^f 	&(A,d) \ar[r]^g \ar[d]^{\eta}_{\simeq} &(C,d) \\
% %    & (B,d) & 
% %   }
% %   }
% %   
% %   where $gf$ is a quasi-isomorphism and $nil \,B \leq m$.
% % %   \item Quasi isomorphic Sullivan algebras have the same category.
% % %  \end{enumerate}
% % \end{Proposition}
% % 
% % \begin{proof}
% % % \begin{enumerate}
% % %  \item 
% % 
% %  Let $\Sullivan$ be a Sullivan algebra with $cat \, \Sullivan = m$ then the diagram 
% % 
% %   \centerline{
% %     \xymatrix{
% %     \Sullivan \ar[r]^(.35){\lambda_{m}} &
% %     (\Lambda V \otimes \Lambda Z(m) , d)
% %     \ar[d]^{\zeta_m}_{\simeq} \ar[r]^(.6){\pi_m} 	& \Sullivan\\
% %     & (\Lambda V / \Lambda^{> m} V , d) & 
% %     }
% %   }
% % (where we used the maps from \ref{def:LSCategory}) fulfills the desired properties.
% % For the other direction see \cite{Felix2001} p.385.
% 
% % \item Let $\varphi \colon \Sullivan \to (\Lambda W,d)$ be a quasi-isomorphism of Sullivan algebras. Let us first 
% % assume that $cat \, \Sullivan = n$ and $cat \, (\Lambda W,d) = m$ for $n,m \neq \infty$.
% % Then we have two commuting diagrams:
% % 
% %  \xymatrix {
% %   \Sullivan \ar[r]^(.4){\lambda_m \varphi} & (\Lambda W \otimes Z(m),d)  
% %   \ar[r]^-{\pi_m} \ar[d]^{\zeta_m}_{\simeq} & (\Lambda W ,d) & \text{and} \\
% %   & (\Lambda W / \Lambda^{> m} W , d) & &  
% %  }
% % 
% % \end{enumerate}
%  
% \end{proof}
% %TODO nochmal über Corrollary 1 p.385 nachdenken und hoffentlich hier rein bringen

\begin{Definition}
 If $\Sullivan$ is a Sullivan algebra with $V$ finite dimensional, $d(V^{even}) = 0$,
 and $d(V^{odd}) \subseteq \Lambda V^{even}$ we call $\Sullivan$ a \emph{pure Sullivan algebra}.
\end{Definition}

In the following we use $ Q \coloneqq V^{even}$ and $P \coloneqq V^{odd}$.
If $\Sullivan$ is pure the requirements on the differential induce a sequence

\centerline{
\xymatrix{
\cdots \ar[r]^(.35)d &\Lambda Q \otimes \Lambda^2 P \ar[r]^(.55)d &Q \otimes P \ar[r]^d & \Lambda Q \ar[r]^(.55)d &0 
}
}

which can be regarded as a differential graded module $(M,d)$ with $0$ in degrees $\geq 1$, $\Lambda Q$ in degree $0$,
$Q \otimes P$ in degree $-1$ and so on.
We denote the homology of this sequence by $H_k(\Lambda V,d) \coloneqq H^{-k}(M,d)$. In particular, this yields
$$ H(\Lambda V,d) = \bigoplus_k H_k(\Lambda V,d).$$
For an element $x \otimes y \in \Lambda Q \otimes P$ we have 
$ d( x \otimes y) = dx \otimes y + x \otimes dy = x \otimes dy$. This shows
$d( \Lambda Q \otimes P) = d(P) \cdot \Lambda Q$ and further
$$H_0(\Lambda V, d) \cong \Lambda Q / \Lambda Q \cdot d(P)$$

\begin{Proposition}
\label{prop:FiniteDimensionDependentOnDegreeOne}

 For a pure Sullivan algebra $\Sullivan$ the following are equivalent:
 
 \begin{enumerate}
  \item\label{itm:PropositionItmFirst} $H(\Lambda V,d)$ is finite dimensional.
  \item\label{itm:PropositionItmSecond}$H_0(\Lambda V,d)$ is finite dimensional.
 \end{enumerate}

\end{Proposition}

\begin{proof}
 That~\ref{itm:PropositionItmFirst}) implies~\ref{itm:PropositionItmSecond}) follows
 from the remark above. \par Consider that $H_0(\Lambda V,d)$ is finite dimensional.
 We will show that $H(\Lambda V, d)$ is finitely generated as $H_0 \Sullivan$-module which implies~\ref{itm:PropositionItmFirst}).
 First observe that $\Lambda Q$ is a polynomial $\Q$-algebra in finitely many generators and thus Noetherian by
 Hilbert's basis theorem (\ref{thm:HilbertBasisTheorem}). This implies that $\Lambda V$ is a Noetherian $\Lambda Q$-module since it is 
 finitely generated by all products of basis elements in $ P$. Next, observe that
 $ker \, d$ is a $(\Lambda Q)$-module, since for $x \in ker \, d$ and  $y \in \Lambda Q$ we have
 $$ d(x \cdot y) = dx \cdot y + x \cdot dy = 0 + 0 = 0$$
 using $d(\Lambda Q) = 0$. Therefore, $\ker \, d$ is a $(\Lambda Q)$-submodule of $\Lambda V$ and thus 
 finitely generated by some elements $x_1, \ldots, x_n$. Taking homology yields that $[x_1], \ldots, [x_n]$
 are generators of $H \Sullivan$ as $H_0 \Sullivan$-module.
\end{proof}

Given this we have a simple criterion to check if a pure Sullivan algebra has finite dimensional
homology. 
Since not all Sullivan algebras are pure we  now introduce a technique that converts an ordinary Sullivan algebra
into a pure one. Then we shall see that the converted algebra has finite dimensional homology if and only if the original one had.

\begin{Definition}
 Let $\Sullivan$ be a Sullivan algebra with $V = P \oplus Q$ (where $Q = V^{even}$ and $P = V^{odd}$)
 finite dimensional. We define its 
 \emph{associated pure Sullivan algebra} $(\Lambda V, d_{\sigma})$ by:
  
 \begin{align*}\restr{d_{\sigma}}{(\Lambda Q \otimes \Lambda^k P)^p} \coloneqq \pi_{
 (\Lambda Q \otimes \Lambda^{k-1} P)^{p+1}} \circ d 
 & &\text{with} \quad \pi_A \colon \Lambda V \to A \quad \text{being the projection} \\
 \end{align*}
 where formally $(\Lambda Q \otimes \Lambda^{-1} P) \coloneqq 0$. 
 This $d_{\sigma}$ defines a differential (see~\cite{Felix2001} p.438). Further, $(\Lambda V, d_{\sigma})$ 
 is a Sullivan algebra since by definition $d_{\sigma}$ preserves the filtration of the 
 nilpotence condition. Moreover, it satisfies:
 \begin{align*}
  d_{\sigma}(\Lambda Q) = 0 & & d_{\sigma}(\Lambda P) \subseteq \Lambda Q & & 
  d - d_{\sigma} \colon P \to \Lambda Q \otimes \Lambda P^{> 0}
 \end{align*}
  which implies that $(\Lambda V, d_{\sigma})$ is always pure.
 
 \end{Definition}

 \begin{Remark}
  If $\Sullivan$ is a pure Sullivan algebra then $d_{\sigma} = d$.
 \end{Remark}

 
 \begin{Example}
 \label{ex:AssociatedNotMinimalSullivanAlgebra}
  Let us see how the construction of the assiocated pure Sullivan algebra works at an example. 
  Consider the Sullivan algebra $$\Sullivan \coloneqq \Lambda( a_3, b_2 ; db = a)$$ which is clearly not pure since
  $b \in V^{even}$ and $db = a \in V^{odd}$. Thus, we will have $d \neq d_{\sigma}$.
  Further, $d_{\sigma}$ is completely determined by
  its image on $a$ and $b$. Since $da = 0$ we have $d_{\sigma} (a) = 0$. Furthermore,
  $b \in (\Lambda Q \otimes \Lambda^0 P)^2$ and therefore 
  $$d_{\sigma} (b) = (\pi_{ (\Lambda Q \otimes \Lambda^{-1} P)^{3}} \circ d) (b) = \pi_0 (a) = 0$$
  Hence, we obtain $(\Lambda V, d_{\sigma}) \cong \Lambda(a_3, b_2;)$.
  \end{Example}

 
 \begin{Proposition}
\label{prop:EquivalenceFiniteDimensionCategoryCohomology}
 For a minimal Sullivan algebra $\Sullivan$ with $V = V^{\geq 2}$ finite dimensional the following
 are equivalent:
 
 \begin{enumerate}
  \item $H(\Lambda V, d_{\sigma})$ is finite dimensional.
  \item $H(\Lambda V, d)$ is finite dimensional.
  \item The category of $\Sullivan$ is finite.
 \end{enumerate}

\end{Proposition}

\begin{proof}
  %TODO Hier nach Möglichkeit noch 2 -> 3 einfügen, wenn 29.3 verstanden wurde
 See~\cite{Felix2001} Proposition 32.4.
\end{proof}

If a Sullivan algebra $\Sullivan$ satisfies the conditions of~\ref{prop:EquivalenceFiniteDimensionCategoryCohomology},
we can use its associated pure Sullivan algebra $(\Lambda V, d_{\sigma})$ and criterion~\ref{prop:FiniteDimensionDependentOnDegreeOne}
to check if $\Sullivan$ has finite dimensional homology and category. Most of the times this will be easier than 
a direct proof of these properties.

\begin{Remark}
 As we can see from Example~\ref{ex:AssociatedNotMinimalSullivanAlgebra} Proposition 
~\ref{prop:EquivalenceFiniteDimensionCategoryCohomology} does not hold if $\Sullivan$ is not minimal as follows: \\
 The Sullivan algebra $$\Sullivan \coloneqq \Lambda( a_3, b_2 ; db = a)$$ is not minimal and 
 has no homology above degree $0$ (since
 the only element that could generate homology is $a$ but $a = db$ and hence $[a] = 0$). In contrary, the associated
 pure Sullivan algebra $ (\Lambda V, d_{\sigma}) \cong \Lambda (a_3, b_2 ;)$ has infinite dimensional homology which is 
 generated by the elements
 $[b^i]$ and $[ab^i]$ for $i \in \N$.
\end{Remark}
