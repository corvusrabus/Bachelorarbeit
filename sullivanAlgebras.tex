
\section{Sullivan algebras}
As we now introduce the algebraic tools for rational homotopy theory we want the ground ring of all modules and algebras
considered in this section to be $\Q$.

\begin{Definition}
 Given a free graded module $V = {\lbrace {V^i}\rbrace}_{ i \geq 1} $ we call $(\Lambda V, d)$ a \emph{Sullivan algebra} 
 if it satisfies the following properties :
 
  There exists a filtration of graded subspaces $V(0) \subseteq V(1) \subseteq V(2) \subseteq \cdots $ of $V$
  with $\bigcup_{i \in \N_0} V(i) = V$ such that 
  $$ d(V(0)) = 0 \; \text{and} \; d( V(k)) \subseteq  \Lambda V(k-1) $$
  
 This property is called the \emph{nilpotence} property.
 Furthermore, we call a Sullivan Algebra $(\Lambda V,d)$ \emph{minimal} if $im(d) \subseteq \Lambda^{\geq 2} V$
\end{Definition}

Note that the structure of a Sullivan algebra $\Sullivan$ is completely encoded by the graded module $V$ and its
differential $d$. That is, if we are given a differential (which is a derivation) $d \colon V \to \Lambda V$ then the
universal property of free commutative algebras tells us that $d$ extends uniquely to $d \colon \Lambda V \to \Lambda V$.
Therefore it is enough to specify the differential of a Sullivan algebra $\Sullivan$ on the basis of $V$.
To get a better understanding of this structure we will now present some examples.

\begin{Example}
 First we want to present an algebra which is not a Sullivan algebra. 
 Consider the graded $\Q$-vector space $V = \langle x,y,z \rangle$ with $|x| = |y| = |z| = 1$ and 
 equip it with a differential $d$ which is defined by $dx = yz$ , $dy = xz$, $dz = xy$. \newline
 If $\Sullivan$ was a Sullivan algebra it would be minimal since 
 the differential only hits elements of wordlength greater $2$. But this tells us that we could choose a filtration 
 ${\lbrace V(i) \rbrace}_{i \geq 0}$ for
 the nilpotence condition which respects the degree of elements, hence $ x $ would be in some $V(i)$ and $yz$ in some $V(j)$
 with $i \leq j$ but this contradicts $dx \in \Lambda V(i-1)$ ($x$ can not be in $V(0)$ since it is not a cocycle).
 
\end{Example}

Next we want to show that Sullivan algebras provide ``good understandable models'' for the homology of commutative cochain algebras. That motivates
the following definition:

\begin{Definition}
  A \emph{Sullivan model} of a commutative cochain algebra $(A,d)$ is a Sullivan algebra $\Sullivan$ with a quasi-isomorphism
  $\varphi \colon \Sullivan \overset{\simeq}{\to} (A,d)$. If $\Sullivan$ is minimal we also speak of a 
  \emph{minimal Sullivan model}.
\end{Definition}

\begin{Remark}
\label{rem:MinimalSullivanModelsExist}
This definition becomes interesting since it can be shown that every connected commutative cochain algebra has a minimal Sullivan model
(\cite{Felix2001} Theorem 14.12). However, we want to prove a simpler version that will suffice for later use and can
be algorithmically obtained.
\end{Remark}

\begin{Theorem}
 Every simply connected commutative cochain algebra has a minimal Sullivan model.
\end{Theorem}

In the case of the last theorem the model can be obtained by an algorithm which we will use to prove the theorem.

\begin{Algorithm}
 Suppose $(A,d)$ is a simply connected cochain algebra, we inductively construct a Sullivan algebra $\Sullivan$ and a morphism
 $\varphi \Sullivan \to (A,d)$ as follows:
 
 \begin{enumerate}
  \item Choose a basis $([v_i])_{i \in I}$ of $H^2(A,d)$, define $V^2 \coloneqq \langle (\alpha_i)_{i \in I} \rangle$ and
  $\varphi_2 \colon (\Lambda V^2,0) \to (A,d)$ by $\varphi_2(\alpha_i) = v_i$. This clearly induces an isomorphism 
  $H^2(\varphi_2) \colon H^2(\Lambda V^2, 0) \overset{\cong}{\to} H^2(A,d)$.
   
   We now proceed inductively and assume that we have constructed (for $k \geq 2$) $(V^{\leq k}, d)$ and
   ${\varphi_k \colon (\Lambda V^{\leq k}, d) \to (A,d)}$ which is a homology isomorphism in degrees $\leq k$.
   Next we want to extend this to a map $\varphi_{k+1} \colon (\Lambda V^{k+1}, d) \to (A,d)$ with 
   ${\varphi_{k+1}}_{|V^k} = \varphi_k$, $H^{k+1}(\varphi_{k+1})$ bijective and $H^{k + 2}(\varphi_{k+1})$ injective.
   
   \item \label{itm:first} Choose elements $(v_i)_{i \in I} \subset A^{k+1}$ such that
   
   $$H^{k+1}(A,d) = im \; H^{k+1}(\varphi_k) \; \oplus \; \bigoplus_{i \in I} \; \langle [v_i] \rangle $$
   
   \item Further choose cocycles $(c_j)_{j \in J} \subset (\Lambda V^{\leq k})^{k+2}$ with
   $$ ker \; H^{k+2}(\varphi_k) = \bigoplus_{j \in J} \; \langle [c_j] \rangle$$
   
   \item By construction the $(\varphi_k(c_j))_{j \in J}$ are coboundaries and we can choose 
   $(w_j)_{j \in J} \subset A^{k+1}$ with $\varphi_k (c_j) = d(w_j)$.
   \item Now let 
   $V^{k+1}$ be the vector space generated by the formal basis $(\alpha_i)_{i \in I} \cup (\beta_j)_{j \in J}$. We extend $d$
   to a derivation in $\Lambda V^{\leq k+1}$ by $d \alpha_i = 0$ and $d \beta_j = c_j$. Moreover, we extend $\varphi_k$
   to $\varphi_{k+1} \colon \Lambda V^{ \leq k+1} \to A$ by $\varphi_{k+1} (\alpha_i) = v_i$ and 
   $\varphi_{k+1} (\beta_j) = w_j$. Now go to \ref{itm:first} and repeat the process.
 \end{enumerate}

\end{Algorithm}


The next definition is about homotopy groups of a Sullivan algebra, of course they bear that name since we will later
associate them with homotopy groups of a topological space.

\begin{Definition}
 Let $\Sullivan$ be a Sullivan algebra, we define its complex of \emph{indecomposables} as
 $(Q(\Lambda V),\bar{d}) \coloneqq (\Lambda V / \Lambda^{\geq 2} V , \bar{d})$ where $\bar{d}$ is the induced differential.
 Given this we define the \emph{(dual) homotopy groups} of $\Sullivan$ by 
 $$ \pi^n ( \Sullivan) \coloneqq H^n(Q(\Lambda V),\bar{d}) \quad \text{for $n \geq 1$}$$
\end{Definition}

\begin{Remark}
 If $\Sullivan$ is a minimal Sullivan algebra then its homotopy groups can be obtained fairly simple. The minimality property
 implies that the differential on the complex of indecomposables of $\Sullivan$ is zero and hence taking
 homology does not change anything. Therefore, 
 $$ \pi^n ( \Sullivan) = H^n(Q(\Lambda V),0) \cong V^n$$
\end{Remark}
