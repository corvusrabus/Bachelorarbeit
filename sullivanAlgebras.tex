\section{Sullivan algebras}

One of the nicest things about rational homotopy theory is that we can assign to every space a (up to a
suited isomorphism) unique algebraic model in form of a so called \emph{Sullivan algebra}. This theory
will be sketched in section \ref{sec:FromSpacesToAlgebras}. The models will contain a lot of information
about the accompanying space and they also have the property that some subclasses of them can be
represented in a computer which will allow us to make statements about the computational
complexity of certain topological problems.


\subsection{Definition}
Although most definitions and Theorems in this section would work fine for a general ground ring $R$, we now want the
ground ring of all modules and algebras considered in this section to be $\Q$.

\begin{Definition}
 Given a free graded module $V = {\lbrace {V^i}\rbrace}_{ i \geq 1} $ we call $(\Lambda V, d)$ a \emph{Sullivan algebra} 
 if it satisfies the following properties :
 
  There exists a filtration of graded subspaces $V(0) \subseteq V(1) \subseteq V(2) \subseteq \cdots $ of $V$
  with $\bigcup_{i \in \N_0} V(i) = V$ such that 
  $$ d(V(0)) = 0 \; \text{and} \; d( V(k)) \subseteq  \Lambda V(k-1) $$
  
 This property is called the \emph{nilpotence} property.
 Furthermore, we call a Sullivan Algebra $(\Lambda V,d)$ \emph{minimal} if $im(d) \subseteq \Lambda^{\geq 2} V$
\end{Definition}

%TODO diese Erweiterung ist schlecht formuliert.
Note that the structure of a Sullivan algebra $\Sullivan$ is completely encoded by the graded module $V$ and its
differential $\restr{d}{V}$. That is, if we are given a differential $d \colon V \to \Lambda V$ then the
universal property of free commutative algebras tells us that $d$ extends uniquely to a derivation $d \colon \Lambda V \to \Lambda V$.
Therefore it is enough to specify the differential of a Sullivan algebra $\Sullivan$ on a basis of $V$.


\begin{Proposition}
\label{prop:WellBehavedFiltrations}
 Let $\Sullivan$ be a differential graded algebra with $V = {\lbrace V^n \rbrace}_{n \geq 2}$ and 
 ${im \;d \subseteq \Lambda^{\geq 2} V}$ then $\Sullivan$ is a minimal Sullivan algebra and
 the filtration can be chosen such that $V(i) = V^{i+2} \oplus V(i-1)$ (with $V(-1) = 0$).
\end{Proposition}
\begin{proof}
 The proof of this is mainly a degree argument. We choose the filtration ${V(i) = V^{i+2} \oplus V(i-1)}$ as in the claim
 and show that it satisfies
 the nilpotence condition. Begin with $V(0) = V^2$ and take $v \in V(0)$ then $dv$ lies in
 degree $3$ and $dv \in \Lambda^{\geq 2} V$. But, since $V^0 = V^1 = 0$, it follows that $(\Lambda V)^3$ only contains 
 nonzero elements of wordlength 1, therefore $dv = 0$. 
 It remains to show that for $k \geq 1$, $d(V^{k+2}) \subseteq \Lambda V^{ \leq k+1}$ holds. Take $v \in V^{ k+2}$ then 
 $| dv | = k + 3$ and basically the same argument as before works, if $dv$ contained a term $xy \notin \Lambda V^{\leq k+1}$
 then $|x| = 1$ and $|y| = k + 2$ or $|x| = 0$ and $|y| = k + 3$ but in both cases it holds $x = 0$ thus $xy = 0$.
\end{proof}

To get a better understanding of Sullivan algebras we will now present some examples.

\begin{Example}[A Sullivan algebra which is not minimal]
 \label{ex:AlgebraConstructedFromK3}
 Consider the graded $\Q$-vector space $V \coloneqq \langle a,b,c, x,y,z \rangle$ with $|a| = |b| = |c| = 2$,
 $|x| = |y| = |z| = 1$. Define a map $d \colon V \to \Lambda V$ by 
 \begin{align*}
 da = db = dc = 0 & &
 dx = a + b & & dy = b + c & & dz = c + a.  
 \end{align*}

 
 Let us exemplary check on one generator 
 that $d$ indeed induces a differential: we have $d^2(x) = d(a + b) = da + db = 0$ and the others are similar.
 Next we can choose the filtration $V(0) = \langle a,b,c \rangle$, $V(i) = V$ for $i \geq 1$ and it is 
 immediate that this satisfies the nilpotence condition, hence $\Sullivan$ is a Sullivan algebra. Moreover,
 $\Sullivan$ is not minimal since $dx = a + b$ contains the elements $a$ and $b$ of wordlength $1$. \\
\end{Example}


\begin{Example}[An algebra which is not a Sullivan algebra]
 Consider  $V \coloneqq \langle x,y,z \rangle$ with $|x| = |y| = |z| = 1$ and 
 equip it with a differential $d \colon V \to \Lambda V$ which is defined by $dx = yz$ , $dy = xz$, $dz = xy$. \newline
 First note that the differential is indeed well-defined as can be seen by a little calculation.
 Suppose that $\Sullivan$ defined a Sullivan algebra, this would yield a filtration ${\lbrace V(i) \rbrace}_{i \geq 0}$ satisfying
 the nilpotence condition. Let $i,j$ be the least integers with $x \in V(i)$ and $y \in V(j)$. We get
 $i < j$, since $dx = yz \in \Lambda \langle y,z \rangle$ and $j < i$, since $dy = xz \in \Lambda \langle x,z \rangle$.
 This contradicts the existence of the filtration, hence $\Sullivan$ is not a Sullivan algebra. \\
 Note that we indeed need $3$ generators for such an example since all differential graded algebras of the form
 $\Sullivan$ with $ V = {\lbrace V^i \rbrace}_{i \geq 1}$ and $dim \; V \leq 2$ are Sullivan algebras. This can be 
 seen by checking all cases of possible differentials on $\Sullivan$.
 \end{Example}
 
 Later, we shall be interested if the homology of a Sullivan algebra is finite dimensional and a simple criterion 
 for this is the following:
 \begin{Lemma}
  A free commutative algebra (with differential) $(\Lambda V,d)$ with $V = {\lbrace V_i \rbrace}_{i \geq 1}$ finite dimensional
  has finite dimensional homology (i.e. $H^i \Sullivan$ is finite dimensional for all $i$ and there is $n_0$ with
  $H^n \Sullivan = 0$ for $n \geq n_0$) if and only if for a basis $ v_1, \ldots, v_n$ of $V^{even}$ there is 
  $i \in \N$ with $[v^i_j] = 0$.
 \end{Lemma}
 
 \begin{proof}
  If we choose $n_0$ as in the claim we have that $v^{n_0} \in H^{|v| \, n_0} \Sullivan = 0$ and hence from
  left to right follows.
  Let now $v_1, \ldots, v_n$ be a basis of $V^{even}$, $i \in \N$ such that $v_j^i = 0$ for all $j = 1, \ldots, n$ and let
  $w_1, \ldots, w_m$ be a basis of $V^{odd}$. We know that $\Sullivan$ is generated by finite products
  of $v_1, \ldots, v_n, w_1, \ldots, w_m$  and hence  $H \Sullivan $ is generated by finite 
  products of $[v_1], \ldots, [v_n], [w_1], \ldots, [w_m]$ (since it inherits the multiplicative structure from
  $\Sullivan$). But since $[w_j^2] = 0$ ( the $w_j$ lie in odd degrees) and $[v_j^i] = 0$, there are only finitely
  many nonzero products that can be formed like this and hence $H \Sullivan$ is finite dimensional.
 \end{proof}

 \begin{Example}
 \label{ex:AlgebraConstructedFromK3HasFiniteHomology}
 We can see that the Sullivan algebra $\Sullivan$ from \ref{ex:AlgebraConstructedFromK3} has finite dimensional homology
 as follows: 
 we have 
 $$d \big{(} \frac{x - y + z}{2} \big{)} = \frac{(a + b) - (b + c) + (c + a)}{2} = a \in im \, d$$
 and hence $[a] = 0$. A similar calculation shows $b,c \in im \, d$ and therefore
 $[a] = [b] = [c] = 0$. Hence, the last lemma shows that $H \Sullivan$ is finite dimensional.
 \end{Example}

 Actually, the Sullivan algebra from last example is not far away from having infinite dimensional homology.
 We can see this as follows:
 
 \begin{Example}
 \label{ex:AlgebraFromK3WithoutOneEdge}
  Consider the Sullivan algebra $\Sullivan$ given by $V = \langle a,b,c , x,y \rangle$ with 
  $|a| = |b| = |c| = 2$, $|x| = |y| = 1$, $da = db = dc = 0$, $dx = a + b$ and $dy = b + c$.
  Note that this is the Sullivan algebra from last example without the generator $z$.
  We have that $(ker \, d)^2 = \langle [a], [b], [c] \rangle$ and 
  $(im \, d)^2 = \langle (a + b), (b +c) \rangle$. Hence $dim \, ( (ker \, d)^2 ) = dim \, ((im \, d)^2 )+ 1 = 3$ and 
  we can easily see that $c \notin im \,d$, i.e. $[c] \neq 0$. Further, all elements in $im \, d$ are 
  linear combinations of products of $( a + b)$ and $(b +c)$, this gives us $[c^i] \neq 0$ for all $i \in \N$
  and therefore $H \Sullivan$ has infinite dimension.
 \end{Example}

\begin{Remark}
 Now that we have seen that the definition of example Sullivan algebras follows the same structure
 we want to have a brief representation of them using this structure.
 Therefore, we introduce the following notation: \newline
 We write $\Lambda (a_i , b_j, \ldots, x_k ; d x = a)$ to mean the Sullivan algebra
 $(\Lambda \langle a, b, \ldots, x \rangle , d) $ with $|a| = i$, $|b| = j$, $|x| = k$, $d x = a$ and
 $dy = 0 $ for all other generators $y$, i.e. the subscripts denote the degree of the indexed generator and
 we only specify the differential on generators which get not mapped to zero.
\end{Remark}
%TODO Kann auch weg gelassen werden, je nachdem ob ich das nochmal benutze


Next we want to see that Sullivan algebras provide ``good understandable models'' for the homology of 
commutative cochain algebras. For this we need:

\begin{Definition}
  A \emph{Sullivan model} of a commutative cochain algebra $(A,d)$ is a Sullivan algebra $\Sullivan$ with a quasi-isomorphism
  $\varphi \colon \Sullivan \overset{\simeq}{\to} (A,d)$. If $\Sullivan$ is minimal we also speak of a 
  \emph{minimal Sullivan model}.
\end{Definition}

\begin{Example}
\label{ex:MinimalModelOfSpheres}
 To become familiar with this concept let us compute a minimal Sullivan model of the cohomology 
 algebra $H^*(S^n) \cong (\Q [\alpha] / (\alpha^2), 0)$ (with $|\alpha| = n$) of the
 $n$-sphere. If $n$ is odd this is fairly easy since we can define 
 $\varphi \colon \Lambda (v_n; dv = 0) \to H^*(S^n) $ by $v \mapsto \alpha$. This defines a quasi-isomorphism since $n$ being odd implies
 $ v^2 = 0$. \\
 For $n$ even we have the problem that $v^2 \neq 0$ . Therefore, also 
 $[v^2] \neq 0$ and for the same $\varphi$ as before we would get
 $(H^{2n}(\varphi)) ([v^2]) = 0$ and $\varphi$ would not be quasi-isomorphism since $H(\varphi)$ is not injective.
 Thus, we have to ``kill'' $v^2$
 in homology by introducing a new element that maps to $v^2$. Therefore, define
 $\varphi \colon \Lambda(v_n, x_{2n -1}; dx = v^2) \to H^*(S^n) $ by $v \mapsto \alpha$ and $ x \mapsto 0$.
 This now defines a quasi-isomorphism. Since both constructed algebras are minimal, they define minimal models
 of the corresponding cohomology algebras.
\end{Example}

\begin{Remark}
\label{rem:MinimalSullivanModelsExist}
It can be shown that every connected commutative cochain algebra has a minimal Sullivan model
(\cite{Felix2001} Theorem 14.12). However the proof of this is not constructive and we want to prove a weaker 
version that can be algorithmically obtained and will suffice for later use .
\end{Remark}

\begin{Theorem}
 \label{thm:MinimalSullivanModelsExistForSimplyConnected}
 Every simply connected commutative cochain algebra has a minimal Sullivan model.
\end{Theorem}

In the case of the last theorem the model can be obtained by an algorithm which we will use to prove the theorem.

\begin{Algorithm}
\label{alg:ConstructionOfMinimalSullivanAlgebra}
 Suppose $(A,d)$ is a simply connected cochain algebra; we inductively construct a Sullivan algebra $\Sullivan$ and a morphism
 $\varphi \colon \Sullivan \to (A,d)$ as follows:
 
 \begin{enumerate}
  \item \label{itm:FirstAlgorithmStep} Choose a basis of cocycles $([v_i])_{i \in I}$ of $H^2(A,d)$, 
  define $V^2 \coloneqq \langle (\alpha_i)_{i \in I} \rangle$ and
  $\varphi_2 \colon (\Lambda V^2,0) \to (A,d)$ by $\varphi_2(\alpha_i) = v_i$. This clearly induces an isomorphism 
  $H^2(\varphi_2) \colon H^2(\Lambda V^2, 0) \overset{\cong}{\to} H^2(A,d)$.
   
   We now proceed inductively and assume that we have constructed $(V^{\leq k}, d)$ (for $k \geq 2$) and
   ${\varphi_k \colon (\Lambda V^{\leq k}, d) \to (A,d)}$ which is a homology isomorphism in degrees $\leq k$.
   Next we want to extend this to a map $\varphi_{k+1} \colon (\Lambda V^{k+1}, d) \to (A,d)$ with 
   ${\varphi_{k+1}}_{|V^k} = \varphi_k$, $H^{k+1}(\varphi_{k+1})$ bijective and $H^{k + 2}(\varphi_{k+1})$ injective:
   
   \item \label{itm:first} Choose elements $(v_i)_{i \in I} \subset A^{k+1}$ such that
   
   $$H^{k+1}(A,d) = im \; H^{k+1}(\varphi_k) \; \oplus \; \bigoplus_{i \in I} \; \langle [v_i] \rangle $$
   
   \item \label{itm:third} Further choose cocycles $(c_j)_{j \in J} \subset (\Lambda V^{\leq k})^{k+2}$ with
   $$ ker \; H^{k+2}(\varphi_k) = \bigoplus_{j \in J} \; \langle [c_j] \rangle$$
   
   \item By construction the $(\varphi_k(c_j))_{j \in J}$ are coboundaries and we can choose 
   $(w_j)_{j \in J} \subset A^{k+1}$ with $\varphi_k (c_j) = d(w_j)$.
   \item \label{itm:second} Let 
   $V^{k+1}$ be the vector space generated by the formal basis $(\alpha_i)_{i \in I} \cup (\beta_j)_{j \in J}$. We extend $d$
   to a differential in $\Lambda V^{\leq k+1}$ by $d \alpha_i = 0$ and $d \beta_j = c_j$. Moreover, we extend $\varphi_k$
   to $\varphi_{k+1} \colon \Lambda V^{ \leq k+1} \to A$ by $\varphi_{k+1} (\alpha_i) = v_i$ and 
   $\varphi_{k+1} (\beta_j) = w_j$. Now go to \ref{itm:first} and repeat the process for $k+1$.
 \end{enumerate}

\end{Algorithm}
Let us collect some properties of this algorithm:
\begin{Proposition}
\label{prop:PropertiesOfAlgorithm}
 Let $(A,d)$ be a simply connected cochain algebra and $m \colon \Sullivan \to (A,d)$ the morphism which
 algorithm \ref{alg:ConstructionOfMinimalSullivanAlgebra} produces for $(A,d)$. Then the following  holds:
 \begin{enumerate}
  \item $m \colon \Sullivan \to (A,d)$ is a minimal Sullivan model for $(A,d)$
  \item Let $r$ be the least integer greater zero with $H^r(A) \neq 0$ then $V^i = 0$ for $1 \leq i < r$ and
    $H^r(m) \colon V^r \overset{\cong}{\to} H^r(A)$
  \item Let $r$ be the least integer with $H^r(A)$ infinite dimensional (or else $\infty$) then
    all $V^k$ for $k < r$ are finite dimensional.
 \end{enumerate}

\end{Proposition}
Note that this Proposition also proves \ref{thm:MinimalSullivanModelsExistForSimplyConnected}.
\begin{proof}
\begin{enumerate}
 \item 

 First observe that $\Sullivan$ actually is a Sullivan Algebra, by construction the filtration $V(i) \coloneqq V^{i+2}$ satisfies
 the nilpotence condition and further the choice of the differential in step \ref{itm:second} implies that $\Sullivan$ is minimal.
 Next we want to see that $m$ is a quasi-isomorphism, for this we show by induction over $k \geq 2$ that
 $H^i(m_k)$ is an isomorphism for $i \leq k$ and injective for $ i = k +1$. The case $k = 2$ is clear since $m_2$ is chosen
 as an isomorphism in \ref{itm:FirstAlgorithmStep}. Now if $k > 2$ it holds by induction hypothesis (since $m_k$ extends
 $m_{k-1}$) that $H^i(m_k)$ is surjective for $i < k$ and by the choice of elements in \ref{itm:first} also for $i = k$.
 Further, $H^i(m_k)$ is injective by induction hypothesis for $i < k$. Additionally, $H^{k+1}(m_k)$ is injective since
 all elements $[z] \in ker \; H^{k+1}(m_k)$ lie in $\Lambda V^{<k}$ (since $V^1 = 0$) and therefore are
 coboundaries by step \ref{itm:third}. At last, consider $[z] \in ker \; H^k (m_k)$, we can write (using the notation used in
 the algorithm)
 $$z = \sum_{i \in I} \alpha_i \lambda_{\alpha_i} + \sum_{j \in J} \beta_j \lambda_{\beta_j} 
 + r \quad \text{with $r \in \Lambda V^{<k}$ }$$
 
 applying the differential to this yields
 $ -dr = d(\sum_{j \in J} \beta_j \lambda_{\beta_j}) = \sum_{j \in J} \lambda_{\beta_j} c_j$
 and taking homology in this equation gives $ [0] = \sum_{j \in J} \lambda_{\beta_j} [c_j]$ and since
 $[c_j] \neq 0$ for all $j \in J$ by the choice of the $c_j$, we get that $\lambda_{\beta_j} = 0$ for all $j \in J$ and 
 therefore $dr = 0$. We also have:
 $$0 = H^{k}(m_k)(z) = (H^k(m_k))( \sum_{i \in I} \alpha_i \lambda_{\alpha_i} ) + (H^k(m_{k-1}))(r)
    = \sum_{i \in I} \lambda_{\alpha_i} [v_i] + (H^k(m_{k-1}))(r) $$
    
and by choice of the $[v_i]$ it follows that $\lambda_{\alpha_i} = 0$ and therefore 
$ [z] = [r] \in H^k ( m_{k+1})$ and by induction hypothesis $[z] = 0$ follows.
%TODO Beweis eventuell noch mal prüfen

 \item Follows immediately from the construction.
 \item Let wlog $r > 2$ then $V^2 \cong H^2(A,d)$ is finite dimensional. Now we can proceed inductively, 
 $V^k$ is by construction a
 direct sum of a subspace of $H^k(A,d)$ and a subspace of $(\Lambda V^{ <k })^k$ and therefore is finite dimensional since
 the latter two are by induction hypothesis and by requirement for $k<r$.
 \end{enumerate}
\end{proof}

Further, the term model of a commutative cochain algebra would be a bad choice if it was not unique up to isomorphism which 
is the next Theorem:

\begin{Theorem}
 \label{thm:UniquenessOfModels}
 If $(A,d)$ is a connected commutative cochain algebra then all minimal Sullivan models of $(A,d)$ are isomorphic
 (as graded algebras).
\end{Theorem}
\begin{proof}
 A proof would require to introduce some theory about homotopy of Sullivan algebras so we refer to \cite{Felix2001}
 p.191 Theorem 14.12.
\end{proof}


The next definition is about homotopy groups of a Sullivan algebra, of course they bear that name since we will later
associate them with homotopy groups of a topological space which gets modeled by the Sullivan model.

\begin{Definition}
 Let $\Sullivan$ be a Sullivan algebra, we define its complex of \emph{indecomposables} as
 $(Q(\Lambda V),\bar{d}) \coloneqq (\Lambda V / \Lambda^{\geq 2} V , \bar{d})$ where $\bar{d}$ is the induced differential.
 Given this we define the \emph{(dual) homotopy groups} of $\Sullivan$ by 
 $$ \pi^n ( \Sullivan) \coloneqq H^n(Q(\Lambda V),\bar{d}) \quad \text{for $n \geq 1$}$$
\end{Definition}

\begin{Remark}
 If $\Sullivan$ is a minimal Sullivan algebra then its homotopy groups can be obtained fairly simple. The minimality property
 implies that the differential on the complex of indecomposables of $\Sullivan$ is zero and hence taking
 homology does not change anything. Therefore, 
 $$ \pi^n ( \Sullivan) = H^n(Q(\Lambda V),0) \cong V^n$$
\end{Remark}

\subsection{Relative Sullivan algebras}

Sullivan algebras are sometimes not enough to describe some aspects and we therefore introduce somehow more sophisticated
Sullivan algebras, called relative Sullivan algebras.

\begin{Definition}
 We call $(B \otimes \Lambda V,d)$ a \emph{relative Sullivan algebra} if it satisfies the following:
 
 \begin{itemize}
  \item $(B,d) \coloneqq (B \otimes 1, d)$ is a a connected sub cochain algebra.
  \item $1 \otimes V = V = { \lbrace V^p \rbrace}_{ p \geq 1}$ is a graded vector space.
  \item There is a filtration $V(0) \subseteq V(1) \subseteq V(2) \subseteq \ldots$ of graded subspaces
    with  $ V = \bigcup_{k = 0}^{\infty} V(k)$ such that 
    \begin{align*}
     d(V(0)) \subseteq B &  & \text{and} & & d(V(k)) \subseteq B \otimes \Lambda (V(k-1)) \quad \text{for $k \geq 1$.}
    \end{align*}
 \end{itemize}
 Moreover, $(B \otimes \Lambda V,d)$ is called \emph{minimal} if 
 $im \, d \subseteq B^{\geq 1} \otimes \Lambda V + B \otimes \Lambda^{\geq 2} V$.
\end{Definition}

\begin{Remark}
 Note that for $B = \Q$ the definition of Sullivan algebra and relative Sullivan algebra are the same.
\end{Remark}

\begin{Definition}

 A \emph{Sullivan model} for a morphism $\varphi \colon (B,d) \to (C,d)$ of commutative cochain algebras is
 a relative Sullivan algebra $\RelSullivan$ and a quasi-isomorphism
 $$ m \colon \RelSullivan \overset{\simeq}{\to} (C,d) $$
 such that $\restr{m}{B \otimes 1} = \varphi$ 
\end{Definition}

\begin{Proposition}
\label{prop:ExistenceMorphismSullivanModels}
 A morphism $\varphi \colon (B,d) \to (C,d)$ of commutative cochain algebras has a Sullivan model
 if $H^0(B) = \Q = H^0(C)$ and $H^1(\varphi)$ is injective.
\end{Proposition}
\begin{proof}
 Refer to \cite{Felix2001} (Proposition 14.3).
\end{proof}

\subsection{Properties of Sullivan Algebras}
Here we want to introduce some notations for properties of Sullivan algebras. It will occur that it is not 
immediately clear why some properties of Sullivan algebras could be interesting and why we give them a name. But, we assure
the reader that we shall later see that all these definitions correspond 1:1 to topological properties which
get modeled by Sullivan algebras.

\par

Let $\Sullivan$ be a Sullivan algebra and consider the projection 
$\varrho_m \colon \Sullivan \to (\Lambda V / \Lambda^{> m} V , d)$ for $m \geq 1$, then $H^1(\varrho_m)$ is injective.
Therefore, Proposition \ref{prop:ExistenceMorphismSullivanModels} tells us that there is a Sullivan model

\centerline{
\xymatrix{
\Sullivan \ar[r]^(.35){\lambda_{m}} \ar[rd]_{\varrho_m} &
(\Lambda V \otimes \Lambda Z(m) , d)
\ar[d]^{\zeta_m}_{\simeq} \\
 & (\Lambda V / \Lambda^{> m} V , d)
}
}

of $\varrho_m$ (where $Z(m)$ is just a notation for the Sullivan algebra which models $\varrho_m$ in dependence of $m$).

\begin{Definition}
\label{def:LSCategory}
 The \emph{(Lusternik-Schnirrelmann) category} of a Sullivan Algebra $\Sullivan$ is (using the notation from above) the 
 least integer $m$ (or $\infty$) such that there exists a morphism 
 $\pi_m \colon (\Lambda V \otimes \Lambda Z(m) , d) \to \Sullivan$ with $\pi_m \circ \lambda_m = id_{\Lambda V}$.
 It will be denoted by $cat \, \Sullivan$.
\end{Definition}

\begin{Definition}
 The \emph{product length} of a cochain algebra $A$ is the smallest integer $n$ (or $\infty$) such that 
 $\Pi_{i = 1}^{n+1} a_i = 0$  for all $a_i \in A$. It will be denoted by $nil \,A$.
\end{Definition}

\begin{Example}
 The product length of $\Lambda \langle x \rangle$ is 2 for $|x|$ odd and $\infty$ for $|x|$ even
 (see \ref{ex:FreeCommutativeEvenDegrees} and \ref{ex:FreeCommutativeOddDegrees}).
\end{Example}

\begin{Proposition}
%  \begin{enumerate}
%   \item 
  A Sullivan algebra $\Sullivan$ has category $ \leq m$ if and only if there exists a diagram of commutative cochain
  algebras 
  
  \centerline{
  \xymatrix{
  \Sullivan \ar[r]^f 	&(A,d) \ar[r]^g \ar[d]^{\eta}_{\simeq} &(C,d) \\
   & (B,d) & 
  }
  }
  
  where $gf$ is a quasi-isomorphism and $nil \,B \leq m$.
%   \item Quasi isomorphic Sullivan algebras have the same category.
%  \end{enumerate}
\end{Proposition}

\begin{proof}
% \begin{enumerate}
%  \item 

 Let $\Sullivan$ be a Sullivan algebra with $cat \, \Sullivan = m$ then the diagram 

  \centerline{
    \xymatrix{
    \Sullivan \ar[r]^(.35){\lambda_{m}} &
    (\Lambda V \otimes \Lambda Z(m) , d)
    \ar[d]^{\zeta_m}_{\simeq} \ar[r]^(.6){\pi_m} 	& \Sullivan\\
    & (\Lambda V / \Lambda^{> m} V , d) & 
    }
  }
(where we used the maps from \ref{def:LSCategory}) fulfills the desired properties.
For the other direction see \cite{Felix2001} p.385.

% \item Let $\varphi \colon \Sullivan \to (\Lambda W,d)$ be a quasi-isomorphism of Sullivan algebras. Let us first 
% assume that $cat \, \Sullivan = n$ and $cat \, (\Lambda W,d) = m$ for $n,m \neq \infty$.
% Then we have two commuting diagrams:
% 
%  \xymatrix {
%   \Sullivan \ar[r]^(.4){\lambda_m \varphi} & (\Lambda W \otimes Z(m),d)  
%   \ar[r]^-{\pi_m} \ar[d]^{\zeta_m}_{\simeq} & (\Lambda W ,d) & \text{and} \\
%   & (\Lambda W / \Lambda^{> m} W , d) & &  
%  }
% 
% \end{enumerate}
 
\end{proof}
%TODO nochmal über Corrollary 1 p.385 nachdenken und hoffentlich hier rein bringen

\begin{Definition}
 If $\Sullivan$ is a Sullivan algebra with $V$ finite dimensional, $d(V^{even}) = 0$
 and $d(V^{odd}) \subseteq \Lambda V^{even}$ we call $\Sullivan$ a \emph{pure Sullivan algebra}.
\end{Definition}

For a Sullivan algebra $\Sullivan$ let us denote $ Q \coloneqq V^{even}$ and $P \coloneqq V^{odd}$.
If $\Sullivan$ is pure then the requirements on the differential induce a sequence

\centerline{
\xymatrix{
\cdots \ar[r]^(.35)d &\Lambda Q \otimes \Lambda^2 P \ar[r]^(.55)d &Q \otimes P \ar[r]^d & \Lambda Q \ar[r]^(.55)d &0 
}
}

which we can regard as a differential graded module $(M,d)$ with $0$ in degrees $\geq 1$, $\Lambda Q$ in degree $0$,
$Q \otimes P$ in degree $-1$ and so on.
We denote the homology of this sequence by $H_k(\Lambda V,d) \coloneqq H^{-k}(M,d)$. In particular, this yields
$ H(\Lambda V,d) = \bigoplus_k H_k(\Lambda V,d)$. \par
For an element $x \otimes y \in \Lambda Q \otimes P$ we have 
$ d( x \otimes y) = dx \otimes y + x \otimes dy = x \otimes dy$. This shows that
$d( \Lambda Q \otimes P) = d(P) \cdot \Lambda Q$ and further
$$H_0(\Lambda V, d) \cong \Lambda Q / \Lambda Q \cdot d(P)$$

\begin{Proposition}
\label{prop:FiniteDimensionDependentOnDegreeOne}
 For a pure Sullivan algebra $\Sullivan$ the following holds :
 
  $H(\Lambda V,d)$ is finite dimensional if and only if $H_0(\Lambda V,d)$ is finite dimensional.
 
\end{Proposition}

\begin{proof}
 From left to right follows from the remark above, so let us consider that $H_0(\Lambda V,d)$ is finite dimensional.
 We will show that $H(\Lambda V, d)$ is finitely generated as $H_0 \Sullivan$-module and this implies the claim.
 First observe that $\Lambda Q$ is a polynomial ($\Q$)-algebra in finitely many generators and thus Noetherian by
 Hilbert's basis theorem ( \ref{thm:HilbertBasisTheorem}). This implies that $\Lambda V$ is a Noetherian $\Lambda Q$-module since it is 
 finitely generated by all products of basis elements in $ P$. Next, observe that
 $ker \, d$ is a $(\Lambda Q)$-module, since for $x \in ker \, d$ and  $y \in \Lambda Q$ we have
 $$ d(x \cdot y) = dx \cdot y + x \cdot dy = 0 + 0 = 0$$
 using $d(\Lambda Q) = 0$. Therefore, $\ker \, d$ is a $(\Lambda Q)$-submodule of $\Lambda V$ and thus 
 finitely generated by some elements $x_1, \ldots, x_n$. Taking homology yields that $[x_1], \ldots, [x_n]$
 are generators of $H \Sullivan$ as $H_0 \Sullivan$-module.
\end{proof}

Given this we have a simple criterion at hand which allows us to check if a pure Sullivan algebra has finite dimensional
homology. Since not all Sullivan algebras are pure we shall now introduce a technique to convert an ordinary Sullivan algebra
into a pure one which has finite dimensional homology if and only if the original one had.

\begin{Definition}
 Let $\Sullivan$ be a Sullivan algebra with $V = P \oplus Q$ (where $Q = V^{even}$ and $P = V^{odd}$)
 finite dimensional, we define its 
 \emph{associated pure Sullivan algebra} $(\Lambda V, d_{\sigma})$ by :
  
 \begin{align*}\restr{d_{\sigma}}{(\Lambda Q \otimes \Lambda^k P)^p} \coloneqq \pi_{
 (\Lambda Q \otimes \Lambda^{k-1} P)^{p+1}} \circ d 
 & &\text{with} \quad \pi_A \colon \Lambda V \to A \quad \text{being the projection} \\
 \end{align*}
 and we formally define $(\Lambda Q \otimes \Lambda^{-1} P) \coloneqq 0$. 
 This indeed defines a differential (see  \cite{Felix2001} p.438). Further, $(\Lambda V, d_{\sigma})$ 
 is also a Sullivan algebra since by definition $d_{\sigma}$ preserves the filtration for the 
 nilpotence condition of $\Sullivan$. Moreover, it satisfies:
 \begin{align*}
  d_{\sigma}(\Lambda Q) = 0 & & d_{\sigma}(\Lambda P) \subseteq \Lambda Q & & 
  d - d_{\sigma} \colon P \to \Lambda Q \otimes \Lambda P^{> 0}
 \end{align*}
  which implies that $(\Lambda V, d_{\sigma})$ is always pure.
 
 \end{Definition}

 \begin{Example}
  \label{ex:AssociatedNotMinimalSullivanAlgebra}
  Let us see how the construction of the assiocated pure Sullivan algebra works at an example. For this,
  consider the Sullivan algebra $\Sullivan \coloneqq \Lambda( a_3, b_2 ; db = a)$ which is clearly not pure since
  $b \in V^{even}$ and $db = a \in V^{odd}$. Thus, the construction of the associated pure Sullivan algebra 
  will change the differential and is completely determined by
  the image of $d_{\sigma}$ on $a$ and $b$. Since $da = 0$ we also have $d_{\sigma} (a) = 0$. Further
  $b \in (\Lambda Q \otimes \Lambda^0 P)^2$ and therefore 
  $$d_{\sigma} (b) = (\pi_{ (\Lambda Q \otimes \Lambda^{-1} P)^{3}} \circ d) (b) = \pi_0 (a) = 0$$
  Hence, we get $(\Lambda V, d_{\sigma}) = \Lambda(a_3, b_2;)$.
  \end{Example}

 
 \begin{Proposition}
\label{prop:EquivalenceFiniteDimensionCategoryCohomology}
 If $\Sullivan$ is a minimal Sullivan algebra with $V = V^{\geq 2}$ finite dimensional then the following
 are equivalent:
 
 \begin{enumerate}
  \item $H(\Lambda V, d_{\sigma})$ is finite dimensional
  \item $H(\Lambda V, d)$ is finite dimensional
  \item The category of $\Sullivan$ is finite
 \end{enumerate}

\end{Proposition}

\begin{proof}
  %TODO Hier nach Möglichkeit noch 2 -> 3 einfügen, wenn 29.3 verstanden wurde
 See \cite{Felix2001} Proposition 32.4.
\end{proof}

Hence, if a Sullivan algebra $\Sullivan$ satisfies the conditions of \ref{prop:EquivalenceFiniteDimensionCategoryCohomology}
we can use its associated pure Sullivan algebra $(\Lambda V, d_{\sigma})$ and criterion \ref{prop:FiniteDimensionDependentOnDegreeOne}
to check if $\Sullivan$ has finite dimensional homology and category. In many cases this will be easier than 
trying to prove these properties directly.

\begin{Remark}
 As we can see from example \ref{ex:AssociatedNotMinimalSullivanAlgebra} Proposition 
 \ref{prop:EquivalenceFiniteDimensionCategoryCohomology} does not hold if $\Sullivan$ is not minimal as follows:
 The Sullivan algebra $\Sullivan \coloneqq \Lambda( a_3, b_2 ; db = a)$ is not minimal and 
 has no homology above degree $0$ (since
 the only element that could generate homology is $a$ but $a = db$ and hence $[a] = 0$) and the associated
 pure Sullivan algebra $\Lambda (a_3, b_2 ;)$ has infinite dimensional homology which is generated by elements
 $[b^i]$ and $[ab^i]$ for $i \in \N$.
\end{Remark}
