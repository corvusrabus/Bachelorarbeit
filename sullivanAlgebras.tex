
\section{Sullivan algebras}

\begin{Definition}
 Given a free graded module $V = {\lbrace {V^i}\rbrace}_{ i \geq 1} $ we call $(\Lambda V, d)$ a \emph{Sullivan algebra} 
 if it satisfies the following properties :
 
  There exists a filtration of graded subspaces $V(0) \subseteq V(1) \subseteq V(2) \subseteq \cdots $ of $V$
  with $\bigcup_{i \in \N_0} V(i) = V$ such that 
  $$ d(V(0)) = 0 \; \text{and} \; d( V(k)) \subseteq  \Lambda V(k-1) $$
  
 This property is called the \emph{nilpotence} property.
 Furthermore, we call a Sullivan Algebra $(\Lambda V,d)$ \emph{minimal} if $im(d) \subseteq \Lambda^{\geq 2} V$
\end{Definition}

Note that the structure of a Sullivan algebra $\Sullivan$ is completely encoded by the graded module $V$ and its
differential $d$. That is, if we are given a differential (which is a derivation) $d \colon V \to \Lambda V$ then the
universal property of free commutative algebras tells us that $d$ extends uniquely to $d \colon \Lambda V \to \Lambda V$.
Therefore it is enough to specify the differential of a Sullivan algebra $\Sullivan$ on the basis of $V$.
To get a better understanding of this structure we will now present some examples.

\begin{Example}
 First we want to present an algebra which is not a Sullivan algebra. 
 Consider the graded $\Q$-vector space $V = \langle x,y,z \rangle$ with $|x| = |y| = |z| = 1$ and 
 equip it with a differential $d$ which is defined by $dx = yz$ , $dy = xz$, $dz = xy$. \newline
 If $\Sullivan$ was a Sullivan algebra it would be minimal since 
 the differential only hits elements of wordlength greater $2$. But this tells us that we could choose a filtration 
 ${\lbrace V(i) \rbrace}_{i \geq 0}$ for
 the nilpotence condition which respects the degree of elements, hence $ x $ would be in some $V(i)$ and $yz$ in some $V(j)$
 with $i \leq j$ but this contradicts $dx \in \Lambda V(i-1)$ ($x$ can not be in $V(0)$ since it is not a cocycle).
 
 
\end{Example}